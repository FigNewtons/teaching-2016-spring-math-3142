\documentclass[letterpaper, twoside, 12pt]{book}
\usepackage{../notes}

\title{MATH 3142 Streaming Lecture | 2016-02-26}
\date{Updated: \today}
\author{Dr. Steven Clontz\\ UNC Charlotte}

\begin{document}

\maketitle



\begin{exercise}[1]
  Let \(\{\vect u_k\}\) be a sequence in \(\mb R^n\) that converges to
  \(\vect u\). Prove the following for all \(\vect v\in\mb R^n\):
  \[\lim_{k\to\infty}\<\vect u_k,\vect v\>=\<\vect u,\vect v\>.\]
\end{exercise}

\bigskip

\begin{solution}\textbf{[F]:}
Let \(\vect u_k=(\vect u_{k_1},\vect u_{k_2},...,\vect u_{k_n})\) and \(\vect v=(\vect v_1,\vect v_2,...,\vect v_n)\). Recall the proposition ``\(\vect u_k=(\vect u_{k_1},\vect u_{k_2},...,\vect u_{k_n})\) and \(\vect u=(\vect u_1,\vect u_2,...,\vect u_n)\) then \(\lim_{k\to\infty}\vect u_k=\vect u\) iff\(\lim_{k\to\infty}\vect u_{k_i}=\vect u_i\) for each i''. Since \({\vect u_k}\) in \(\mb R^n\) converges to the point \(\vect v\). Then \(\lim_{k\to\infty}\vect u_{k_i}=\vect u_i\). Therefore,
\begin{align*}
      \lim_{k\to\infty}\<\vect u_k,\vect v\> &= \<\lim_{k\to\infty}\vect u_k,\vect v\>\\
      &= \<\lim_{k\to\infty}(\vect u_{k_1},\vect u_{k_2},...,\vect u_{k_n}),\vect\>\\
      &= \<(\lim_{k\to\infty}\vect u_{k_1},\lim_{k\to\infty}\vect u_{k_2},...,\lim_{k\to\infty}\vect u_{k_n}),\vect v\>\\
      &= \<(\vect u_1,\vect u_2,...,\vect u_n),\vect v\>\\
      &= \<\vect u,\vect v\>\\
\end{align*}
\end{solution}

\bigskip

\begin{solution} \textbf{[C]:}
Let \(\vect v\) be a point in \(R^n.\) By the definition of the scalar product, \[\<\vect u_k,\vect v\>=(u_{1k}v_1+...+u_{nk}v_n).\] Note that \(\{\vect u_k\}=(u_{1k},...,u_{nk})\) converges to \(\vect u=(u_1,...,u_n).\) It follows that \(\<\vect u_k,\vect v\>\) converges to \(\<\vect u,\vect v\>=(u_1v_1+...+u_nv_n).\) Therefore, \[\lim_{k\to\infty}\<\vect u_k,\vect v\>=\<\vect u,\vect v\>.\]
\end{solution}

\bigskip

\begin{solution} \textbf{[A]:}
By the Componentwise Convergence Theorem, for any index \(0 \leq i \leq n\), \(\lim_{k\to\infty} p_i (\vect u_k) = p_i (\vect u)\). Then, using the properties of limits, we have
\[
\lim_{k\to\infty} \<\vect u_k,\vect v\> = \lim_{k\to\infty} \sum_{i=1}^n p_i (\vect v) p_i (\vect u_k) = \sum_{i=1}^n p_i (\vect v)\lim_{k\to\infty} p_i (\vect u_k) = \sum_{i=1}^n p_i (\vect v) p_i (\vect u) = \<\vect u,\vect v\>
\]
\end{solution}


\newpage


\begin{exercise}[Section 6.1 \#5]
  Suppose that \(f,g:[a,b]\to\mb R\) are bounded, and have the property that
  \(g(x)\leq f(x)\) for all \(x\in[a,b]\).
  \begin{enumerate}[(a)]
    \item For \(P\) partitioning \([a,b]\), show that
          \(L(g,P)\leq L(f,P)\).
    \item Use part (a) to show that
          \(\underline{\int_a^b}g\leq\underline{\int_a^b}f\).
  \end{enumerate}
\end{exercise}
\begin{solution}
  \begin{enumerate}[(a)]
    \item Let \(m(f,i)=\inf\{f(x):x\in[x_{i-1},x_i]\}\) and
      \(m(g,i)=\inf\{g(x):x\in[x_{i-1},x_i]\}\).
      Applying the defintions of
      \[L(f,P)=\sum_{i=1}^n m(f,i)(x_i-x_{i-1})\] and
      \[L(g,P)=\sum_{i=1}^n m(g,i)(x_i-x_{i-1}),\]
      and noting that \(m(g,i)\leq m(f,i)\), we have that:
      \[
        L(g,P)
          =
        \sum_{i=1}^n m(g,i)(x_i-x_{i-1})
          \leq
        \sum_{i=1}^n m(f,i)(x_i-x_{i-1})
          =
        L(f,P)
      .\]
    \item Again, applying definitions:
      \[
        \underline{\int_a^b}g
          =
        \sup\{L(g,P):P\text{ partitions }[a,b]\}
          \leq
        \sup\{L(f,P):P\text{ partitions }[a,b]\}
          =
        \underline{\int_a^b}f
      .\]
  \end{enumerate}
\end{solution}


\begin{exercise}[Section 6.4 \#8]
  Suppose that \(f:[a,b]\to\mb R\) is bounded and continuous except
  at one point \(x_0\in(a,b)\). Prove that \(f\) is integrable.
  (Hint: let \(f_1:[a,x_0]\to\mb R\) and \(f_2:[x_0,b]\to\mb R\)
  be defined by \(f_1(x)=f(x)\) and \(f_2(x)=f(x)\), then apply
  Theorem 6.19.)
\end{exercise}
\begin{solution}
  Let \(f_1:[a,x_0]\to\mb R\) and \(f_2:[x_0,b]\to\mb R\)
  be defined by \(f_1(x)=f(x)\) and \(f_2(x)=f(x)\).
  Since \(f\) is continuous everywhere in \([a,b]\) except for \(x_0\),
  it follows that \(f_1\) is continuous on \((a,x_0)\) and
  \(f_2\) is continuous on \((x_0,b)\). It follows by Thm 6.19
  that \(f_1,f_2\) are both integrable.

  Let \(\{P_n^1\}\) be an Archemedian sequence for
  \(f_1\) on \([a,x_0]\) and
  \(\{P_n^2\}\) be an Archemedian sequence for \(f_2\) on \([x_0,b]\).
  Then \(P_n=P_n^1\cup P_n^2\) is a partition of \([a,b]\).

  We now show that \(f\) is integrable on \([a,b]\)
  by showing that \(\{P_n\}\)
  is Archemedian for \(f\) on \([a,b]\):
    \[
      \lim_{n\to\infty} U(f,P_n)-L(f,P_n)
        =
      \lim_{n\to\infty}
        (U(f,P_n^1)+U(f,P_n^2))
          -
        (L(f,P_n^1)+L(f,P_n^2))
    \]
    \[
        =
      \left(\lim_{n\to\infty} U(f,P_n^1)-L(f,P_n^1)\right)
        +
      \left(\lim_{n\to\infty} U(f,P_n^2)-L(f,P_n^2)\right)
    \]
    \[
        =
      \left(\lim_{n\to\infty} U(f_1,P_n^1)-L(f_1,P_n^1)\right)
        +
      \left(\lim_{n\to\infty} U(f_2,P_n^2)-L(f_2,P_n^2)\right)
        =
      0 + 0 = 0
    .\]
\end{solution}


\begin{exercise}[Not in book.]
  Let \(A\subseteq\mb R^n\). We say \(A\) is \textbf{clopen} if
  it is both an open and closed subset of \(\mb R^n\). Prove that
  the only nonempty clopen subset of \(\mb R\) is \(\mb R\) itself.
  (Hint: let \(x\in A\) and \(R=\{r>0:B_r(x)\subseteq A\}\),
  and then show that \(R\) has no least upper bound.)
\end{exercise}
\begin{solution}
  Let \(x\in A\) and \(R=\{r>0:B_r(x)\subseteq A\}\), where \(A\)
  is a clopen subset of \(\mb R\). Suppose that \(q\) is a least upper
  bound for \(R\). Since for \(0<r<q\) we have that \(B_r(x)\subseteq A\),
  it follows that \(\bigcup_{0<r<q}B_r(x)=B_q(x)\subseteq A\).
  Note that \(B_q(x)=(x-q,x+q)\).

  Note that \(\{x+q(1-\frac{1}{k})\}\) is a sequence of points in
  \((x-q,x+q)\subseteq A\), and
  \[\lim_{k\to\infty}x+q(1-\frac{1}{k})=x+q(1-0)=x+q.\]
  It follows that \(x+q\in A\). Similarly, we may show that \(x-q\in A\).

  Let \(p>0\) satisify both \(B_p(x-q)=(x-q-p,x-q+p)\subseteq A\) and
  \(B_p(x+q)=(x+q-p,x+q+p)\subseteq A\). Note that
  \[
    B_{p+q}(x)
      =
    (x-q-p,x+q+p)
      =
    (x-q-p,x-q+p)
      \cup
    (x-q,x+q)
      \cup
    (x+q-p,x+q+p)
  \]
  \[
      =
    B_p(x-q)
      \cup
    B_q(x)
      \cup
    B_p(x+q)
      \subseteq
    A
  .\]
  So since \(p+q\in R\) and \(p+q>q\), \(q\) is not an upper bound for
  \(R\). Contradiction.

  Let \(y\in\mb R\) be distinct from \(x\).
  Let \(r=|x-y|>0\). Since \(R\) has no upper bound,
  \(2r\in R\). Since \(y\in B_{2r}(x)\subseteq A\), so \(y\in A\).
  This shows that \(\mb R\subseteq A\subseteq \mb R\), so
  \(A=\mb R\).
\end{solution}

\end{document}


