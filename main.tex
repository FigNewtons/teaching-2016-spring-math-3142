\documentclass[letterpaper, twoside, 12pt]{book}
\usepackage{notes}




\title{MATH 3142 Notes | Spring 2016}
\date{Updated: \today}
\author{Daniel Gruszczynski\\ UNC Charlotte}

\begin{document}

\maketitle

This document is a template for you to take notes in my MATH 3142 course.
For your note check grade, you are required to complete all proofs/solutions
for the problems specified. This template will be updated periodically
throughout the course; you are responsible for updating your copy
as the template is updated. See the syllabus for more details.

You should maintain your notes on Overleaf.com and provide me with a
link so I can check on them. I'll give you notice before notes are ``due'';
when they are due I will download a copy myself from Overleaf.

This is not a replacement for the textbook for this course,
\textit{Advanced Calculus} by Patrick M. Fitzpatrick. Many proofs are
outlined in that text, as well as all the relevant definitions and other
results not included in these notes.

A proof is valid if and only if it uses concepts proven previously in
the book. For example, you cannot prove a lemma in Chapter 6 using
a theorem from Chapter 10, but using a proposition from Chapter 4
is allowed.

I hope you enjoy working through these results. Please email me with
any questions.

\noindent| Dr. Steven Clontz \(\<\)sclontz5@uncc.edu\(\>\)
















\setcounter{chapter}{5}
\chapter{Integration: Two Fundamental Theorems}







\section{Darboux Sums: Upper and Lower Integrals}

\begin{definition}
    Let \(n\) be a natural number. We define \([n] = \{1, 2, ..., n\}\). If 
    \(i\) is an index, we write ``for \(i \in [n]\)'' in place of the usual 
    ``for \(1 \leq i \leq n\).''
\end{definition}

\begin{lemma}[6.1]
  Suppose that the function \(f:[a,b]\to\mb R\) is bounded and the numbers
  \(m,M\) have the property that
  \[
    m\leq f(x)\leq M
  \]
  for all \(x\) in \([a,b]\). Then, if \(P\) is a partition of the domain
  \([a,b]\),
  \[
    m(b-a)\leq L(f,P)
      \text{ and }
    U(f,P)\leq M(b-a)
  .\]
\end{lemma}

\begin{proof}
    Let \(P = \{x_{0}, x_{1}, ..., x_{n}\}\) be a partition on \([a,b]\).
    By assumption, \(m\) is a lower bound of \(f([a,b])\). Restricting
    \(f\) to \([x_{i-1}, x_{i}]\), we have \(m \leq m_{i}\) for all \(i \in [n]\)
    since \(m_{i}\) is the infimum of \(f([x_{i-1}, x_{i}])\). Then, by definition,
    \begin{align*}
        L(f, P) &= \sum_{i = 1}^{n} m_{i}(x_{i} - x_{i-1}) \\ 
                &\geq \sum_{i = 1}^{n} m (x_{i} - x_{i-1}) \\
                &= m \sum_{i = 1}^{n} (x_{i} - x_{i-1}) \\
                &= m (b - a) .\\
    \end{align*}
    Similarly, \(M\) is an upper bound of \(f([a,b])\) and so, when restricting \(f\) 
    to \([x_{i-1}, x_{i}]\), we have \(M \geq M_{i}\) since \(M_{i}\) is the supremum of 
    \(f([x_{i-1}, x_{i}])\) (for all \(i \in [n]\)). Hence, we have
    \begin{align*}
        U(f, P) &= \sum_{i = 1}^{n} M_{i}(x_{i} - x_{i - 1}) \\
                &\leq \sum_{i = 1}^{n} M(x_{i} - x_{i - 1}) \\
                &= M \sum_{i = 1}^{n} (x_{i} - x_{i - 1}) \\
                &= M(b - a) .\\
    \end{align*}

    Therefore, \(m(b - a) \leq L(f, P)\) and \(U(f, P) \leq M(b - a) \). 
\end{proof}


\begin{lemma}[6.2, The Refinement Lemma]
  Suppose that the function \(f:[a,b]\to\mb R\) is bounded and that \(P\)
  is a partition of its domain \([a,b]\). If \(P^\star\) is a refinement
  of \(P\), then
  \[
    L(f,P)\leq L(f,P^\star)
      \text{ and }
    U(f,P^\star)\leq U(f,P)
  .\]
\end{lemma}

\begin{proof}
    Let \(P = \{x_{0}, x_{1}, ..., x_{n}\}\) be a partition on \([a,b]\),
    and let \(P^*\) be its refinement. For \(i \in [n]\), define \(P_{i}\) 
    to be the partition on \([x_{i-1}, x_{i}]\) by the points of 
    \(P^*\) inside this interval. Since \(m_{i} \leq f(x)\) for \(x \in [x_{i-1},x_{i}]\),
    applying the previous lemma to the restriction of \(f\) on \([x_{i-1}, x_{i}]\),
    we have \( m_{i}(x_{i} - x_{i-1}) \leq L(f, P_{i})\). It follows that
    \begin{align*} 
        L(f, P) &= \sum_{i=1}^{n} m_{i}(x_{i} - x_{i - 1}) \\
                &\leq \sum_{i = 1}^{n} L(f, P_{i}) \\
                &= L(f, P^*). \\
    \end{align*}

    Likewise, the previous lemma gives us \(M_{i}(x_{i} - x_{i-1}) \geq U(f, P_{i})\).
    Hence,
    \begin{align*}
        U(f, P) &= \sum_{i=1}^{n} M_{i}(x_{i} - x_{i-1}) \\
                &\geq \sum_{i=1}^{n} U(f, P_{i}) \\
                &= U(f, P^*). \\
    \end{align*}
\end{proof}


\begin{lemma}[6.3]
  Suppose that the function \(f:[a,b]\to\mb R\) is bounded and that
  \(P_1,P_2\) are partitions of its domain. Then \(L(f,P_1)\leq U(f,P_2)\).
\end{lemma}

\begin{proof}
    Let \(P = P_1 \cup P_2\) be the common refinement of partitions \(P_1\)
    and \(P_2\). By the Refinement Lemma, \(L(f, P_{1}) \leq L(f, P)\)
    and \(U(f, P) \leq U(f, P_{2})\). Then, since \(L(f, P) \leq U(f, P)\),
    the transitivity of \(\leq\) implies that \(L(f, P_{1}) \leq U(f, P_{2})\).
\end{proof}


\begin{lemma}[6.4]
  For a bounded function \(f:[a,b]\to\mb R\),
  \[
    \underline{\int_a^b} f
      \leq
    \overline{\int_a^b} f
  .\]
\end{lemma}

\begin{proof}
    Let \(P\) be any partition on \([a,b]\). By the previous lemma,
    \(U(f, P) \geq L(f, P')\) for all partitions \(P'\) on \([a,b]\). 
    It follows that
    \[ \underline{\int_a^b} f \leq U(f, P) . \]
    Since \(P\) was arbitrary, the above shows that \(\underline{\int_a^b} f\)
    is a lower bound for all such \(U(f, P)\). Therefore,
    \[ \underline{\int_a^b} f \leq \overline{\int_a^b} f .\]
\end{proof}

\begin{exercise}[2]
  For an interval \([a,b]\) and a positive number \(\delta\),
  show that there is a partition \(P=\{x_i:0\leq i\leq n\}\) of
  \([a,b]\) such that each partition interval \([x_i,x_{i+1}]\)
  of \(P\) has length less than \(\delta\).
\end{exercise}

\begin{solution}
    Let \([a,b]\) be an interval (\(b > a\)) and \(\delta > 0\).
    By the Archimedean property, there exists a natural number
    \(n\) such that \(\frac{\delta}{b - a} > \frac{1}{n}\). It follows that
    we can form partition intervals of equal length \(\frac{b - a}{n}\):
    \begin{align*}
        \delta &> \frac{b - a}{n} \\
               &= \frac{1}{n} \sum_{i=0}^{n - 1} (x_{i + 1} - x_{i}) \\
               &= \frac{1}{n}[ n (x_{i + 1} - x_{i}) ] \\
               &= x_{i + 1} - x_{i} \\
    \end{align*}
\end{solution}


\begin{exercise}[3]
  Suppose that the bounded function \(f:[a,b]\to\mb R\) has the property
  that for each rational number \(x\) in the interval \([a,b]\),
  \(f(x)=0\). Prove that
  \[
    \underline{\int_a^b}f
      \leq
    0
      \leq
    \overline{\int_a^b}f
  .\]
\end{exercise}

\begin{solution}
    Let \(P = \{x_{0}, x_{1}, ... , x_{n}\}\) be an arbitrary partition on \([a,b]\).
    Since \(\mb Q\) is dense in \(\mb R\), \(m_{i} \leq 0\) and \(M_{i} \geq 0\) for
    all \(i \in [n]\). This implies \(L(f, P) \leq 0\) and \(U(f, P) \geq 0\). 
    Consequently, 
    \[ \underline{\int_a^b} f \leq 0 \leq \overline{\int_a^b} f .\]
\end{solution}


\begin{exercise}[6]
  Suppose that \(f:[a,b]\to\mb R\) is a bounded function for which there is
  a partition \(P\) of \([a,b]\) with \(L(f,P)=U(f,P)\). Prove that
  \(f:[a,b]\to\mb R\) is constant.
\end{exercise}

\begin{solution}
    Let \(P\) be the partition where \(L(f, P) = U(f, P)\). Then
    \[ 0 = U(f, P) - L(f, P) = \sum_{i=1}^{n} M_i(x_{i} - x_{i-1}) -
                                \sum_{i = 1}^{n} m_i (x_{i} - x_{i-1}) 
                            = \sum_{i=1}^{n} (M_{i} - m_{i})(x_{i} - x_{i-1}) .\]
    Since \(x_{i} > x_{i-1}\), \((x_{i} - x_{i-1}) > 0\).  Similarly,
    \((M_{i} - m_{i}) \geq 0\). This implies that the term 
    \((M_{i} - m_{i})(x_{i} - x_{i-1})\) is nonnegative, but since 
    the entire sum is zero, we must have \(M_{i} = m_{i}\) for
    all \(i \in [n]\). It follows that \(f\) takes the same value
    within each partition interval, and since 
    \([x_{i - 1}, x_{i}] \cap [x_{i}, x_{i + 1}] = \{x_{i}\}\), \(f\)
    takes the same value for all of \([a,b]\). Therefore, \(f\)
    is constant.
\end{solution}




\section{The Archimedes-Riemann Theorem}


\begin{lemma}[6.7]
  For a bounded function \(f:[a,b]\to\mb R\) and a partition \(P\) of
  \([a,b]\),
  \[
    L(f,P)
      \leq
    \underline{\int_a^b}f\leq\overline{\int_a^b}f\leq U(f,P)
  .\]
\end{lemma}

\begin{proof}
    By definition, \(\overline{\int_a^b} f \leq U(f, P)\) and
    \(L(f, P) \leq \underline{\int_a^b} f\). Then, by Lemma 6.4
    we have \(\underline{\int_a^b} f \leq \overline{\int_a^b} f\).
    The result follows.
\end{proof}


\begin{theorem}[6.8, The Archimedes-Riemann Theorem]
  Let \(f:[a,b]\to\mb R\) be a bounded function. Then \(f\) is integrable on
  \([a,b]\) if and only if there is a sequence \(\{P_n\}\) of partitions
  of the interval \([a,b]\) such that
  \[
    \lim_{n\to\infty}[U(f,P_n)-L(f,P_n)]=0
  .\]
  Moreover, for any such sequence of partitions,
  \[
    \lim_{n\to\infty} L(f,P_n)
      =
    \int_a^b f
      =
    \lim_{n\to\infty} U(f,P_n)
  .\]
\end{theorem}

\begin{proof}
    Suppose \(f\) is integrable on \([a,b]\). Then by definition
    \[ \underline{\int_a^b} f = \int_a^b f = \overline{\int_a^b} f .\]
    For convenience, let \(L = \underline{\int_a^b} f\) and
    \(U = \overline{\int_a^b} f\). Now, for each \(n \in \mb N\), 
    define \(L_n \equiv L - \frac{1}{n}\) and \(U_n \equiv U + \frac{1}{n}\).
    Since \(L\) is the supremum of the lower Darboux sums of \(f\),
    \(L_n\) is not an upper bound of this collection and so there
    exists a partition \(P'\) such that \(L_n < L(f, P')\). By similar
    reasoning, there exists a partition \(P''\) such that
    \(U(f, P'') < U_n \). Define \(P_n = P' \cup P''\) as their common
    refinement. This gives us
    \[ 0 \leq U(f, P_n) - L(f, P_n) < U_n - L_n = 
        \Bigg [ \int_a^b f + \frac{1}{n} \Bigg ] - 
        \Bigg [ \int_a^b f - \frac{1}{n} \Bigg ] = \frac{2}{n}. \]
    Hence,
    \[ \limit [ U(f, P_n) - L(f, P_n) ] = 2 \limit \frac{1}{n} = 0 \]
    and so \(\{P_n\}\) is an Archimedean sequence. 


    Conversely, suppose we had an Archimedean sequence \(\{P_n\}\) so that 
    \[ \limit [ U(f, P_n) - L(f, P_n) ] = 0 .\]
    Because 
    \[L(f, P_n) \leq \underline{\int_a^b} f \leq \overline{\int_a^b} f \leq U(f, P_n) \]
    by Lemma 6.7, we have (by taking the limit), 
    \[ 0 \leq \overline{\int_a^b} f - \underline{\int_a^b} f \leq \limit [U(f, P_n) - L(f, P_n)] = 0 .\]
    Hence \(\underline{\int_a^b} f = \overline{\int_a^b} f \) and so \(f\)
    is integrable.

    Moreover, Lemma 6.7 shows that \( 0 = \limit U(f, P_n) - \overline{\int_a^b} f \)
    and \(0 = \underline{\int_a^b} f - \limit L(f, P_n) \) and so we get
    \[ \limit L(f, P_n) = \underline{\int_a^b} f = \int_a^b f = \overline{\int_a^b} f = \limit U(f, P_n) .\]
\end{proof}


\begin{example}[6.9]
  Show that
  a monotonically increasing function \(f:[a,b]\to\mb R\) is integrable.
\end{example}

\begin{solution}
    Let \(P_n\) be the regular partition on \([a,b]\). Since \(f\) is 
    monotonically increasing, on a partition interval \([x_{i-1}, x_i]\),
    \(M_i = f(x_{i})\) and \(m_i = f(x_{i-1})\). Then
    \begin{align*}
        \limit [U(f, P_n) - L(f, P_n)] &= \limit \Bigg [ \sumi M_{i}(x_i - x_{i-1}) - \sumi m_i (x_i - x_{i-1}) \Bigg ]\\
            &= \limit \Bigg [ \sumi (M_i - m_i)(x_i - x_{i-1}) \Bigg ] \\
            &= \limit \Bigg [ \sumi (f(x_i) - f(x_{i-1})) \frac{b - a}{n} \Bigg ] \\
            &= \limit \frac{b - a}{n} \Bigg [ \sumi (f(x_i) - f(x_{i-1})) \Bigg ] \\
            &= \limit \frac{b - a}{n} (f(b) - f(a)) \\
            &= 0.
    \end{align*}
    Therefore, by Theorem 6.8, \(f\) is integrable on \([a,b]\).
\end{solution}

\begin{example}[6.11]
  Show that \(\int_0^1 x^2\,dx=\frac{1}{3}\).
\end{example}

\begin{solution}
    Since \(f(x) = x^2\) is monotonically increasing on \([0,1]\), 
    \(f\) is integrable by the above example. Let 
    \(P_{n} = \{x_{0}, x_{1}, ..., x_{n}\}\) be the regular partition 
    on \([0,1]\). Then \(x_i = \frac{i}{n}\) and using the fact that
    \(\sum_{i=1}^{n} i^2 = \frac{n(n +1)(2n+1)}{6}\), we get
    \begin{align*}
        \int_0^1 x^2, dx&= \limit U(f, P_n) \\
                        &= \limit \sumi M_i (x_i - x_{i-1}) \\
                        &= \limit \sumi f(x_{i}) (x_i - x_{i-1}) \\
                        &= \limit \sumi \frac{1}{n} \frac{i^2}{n^2} \\
                        &= \limit \frac{1}{n^3} \sumi i^2 \\
                        &= \limit \frac{1}{n^3} \Bigg [ \frac{n(n+1)(2n+1)}{6} \Bigg ]\\
                        &= \limit \frac{2n^2 + 3n + 1}{6n^2} \\
                        &= \frac{1}{3} .\\
    \end{align*}
\end{solution}

\begin{exercise}[4]
  Prove that for a natural number \(n\),
  \[ \sum_{i=1}^n i = \frac{n(n+1)}{2} .\]
  Then use this fact and the Archimedes-Riemann Theorem to show that
  \(\int_a^b x\,dx=(b^2-a^2)/2\).
\end{exercise}

\begin{solution}
    First, we prove the summation holds by induction on \(n\). If \(n = 1\),
    \(\sum_{i=1}^1 i = 1 = \frac{1(2)}{2}\). Assume this identity holds
    for all natural numbers \(k \leq n\) and now consider \(n + 1\). Then
    \begin{align*}
        \sum_{i=1}^{n+1} i &= \sum_{i = 1}^{n} i + (n + 1) \\
                        &= \frac{n(n+1)}{2} + (n + 1) \\
                        &= \frac{n^2 + 3n + 2}{2} \\
                        &= \frac{ (n+1)((n+1) + 1)}{2} \\
    \end{align*}
    and hence the induction is complete.

    We note that \(f(x) = x\) is monotonically increasing on \(\mb R\) and
    consequently integrable. Thus, for a regular partition \(P_n\) on 
    \([a,b]\), we have
    \begin{align*}
        \int_a^b x\,dx &= \limit U(f, P) \\
                       &= \limit \sumi M_i (x_i - x_{i-1}) \\
                       &= \limit \sumi x_i \frac{b - a}{n} \\
                       &= \limit \sumi \Bigg (a + i \frac{b-a}{n} \Bigg ) \frac{b-a}{n} \\
                       &= \limit \frac{b-a}{n} \Bigg [ \sumi a + \frac{b-a}{n} \sumi i \Bigg ] \\
                       &= \limit \frac{b-a}{n} \Bigg [ na + \frac{b-a}{n} \cdot \frac{n(n+1)}{2} \Bigg ] \\
                       &= \limit [(ab - a^2) + \frac{(b-a)^2 (n+1)}{2n}] \\
                       &= ab - a^2 + \frac{(b-a)^2}{2} \\
                       &= \frac{2ab - 2a^2 + b^2 - 2ab + a^2}{2} \\
                       &= \frac{b^2 - a^2}{2} .\\
    \end{align*}
\end{solution}


\begin{exercise}[6b]
  Use the Archimedes-Riemann Theorem to show that for \(0\leq a<b\),
  \[
    \int_a^b x^2\,dx = \frac{b^3-a^3}{3}
  .\]
\end{exercise}

\begin{solution}
    Generalizing from Example 6.11, 
     \begin{align*}
        \int_a^b x^2, dx&= \limit U(f, P_n) \\
                        &= \limit \sumi M_i (x_i - x_{i-1}) \\
                        &= \limit \sumi f(x_{i}) (x_i - x_{i-1}) \\
                        &= \limit \sumi \frac{b - a}{n} (a + \frac{b-a}{n} i)^2 \\
                        &= \limit \frac{b - a}{n} \Bigg [ a^2 \sumi 1 + 2a \frac{b-a}{n} \sumi i + \frac{(b-a)^2}{n^2} \sumi i^2 \Bigg ] \\ 
                        &= \limit \frac{b - a}{n} \Bigg [ na^2 + a(b - a)(n + 1) + \frac{ (n + 1)(2n + 1)(b - a)^2}{6n} \Bigg ] \\
                        &= \limit a^2 (b - a) + \limit a(b - a)^2 \frac{n+1}{n} + \limit \frac{ (2n^2 + 3n + 1) (b - a)^3}{6n^2} \\
                        &= a^2 (b - a) + a(b - a)^2 + \frac{(b - a)^3}{3} \\
                        &= \frac{1}{3} \Bigg [ (3a^2 b - 3a^3) + (3ab^2 - 6a^2 b + 3a^3) + (b^3 - 3ab^2 + 3a^2 b - a^3) \Bigg ] \\  
                        &= \frac{b^3 - a^3}{3} \\
    \end{align*}
\end{solution}


\begin{exercise}[9]
  Suppose that the functions \(f:[a,b]\to\mb R\) and
  \(g:[a,b]\to\mb R\) are integrable. Show that there is a sequence
  \(\{P_n\}\) of partitions of \([a,b]\) that is an Archimediean sequence
  of partitions for \(f\) on \([a,b]\) and also an Archimedean sequence
  of partitions for \(g\) on \([a,b]\).
\end{exercise}
\begin{solution}
    By the Archimedes-Riemann Theorem, there exists Archimedean sequences
    \(Q_n\) and \(R_n\) for \(f\) and \(g\), respectively, such that 
    \( \limit [U(f, Q_n) - L(f, Q_n)] = 0\) and \(\limit [U(g, R_n) - L(g, R_n)] = 0\).
    For each \(n\), define \(P_n = Q_n \cup R_n\). The Refinement lemma implies
    \[ 0 = \limit [U(f, Q_n) - L(f, Q_n)] \geq \limit [U(f, P_n) - L(f, P_n)] \geq 0 \]
    and 
    \[ 0 = \limit [U(g, R_n) - L(g, R_n)] \geq \limit [U(g, P_n) - L(g, P_n)] \geq 0 .\]
    Therefore, \(\{P_n\}\) is an Archimedean sequence for \(f\) and \(g\).
\end{solution}




\section{Additivity, Monotonicity, and Linearity}


\begin{theorem}[6.12, Additivity over Intervals]
  Let \(f:[a,b]\to\mb R\) be integrable on \([a,b]\) and let \(c\in(a,b)\).
  Then \(f\) is integrable on \([a,c]\) and \([c,b]\), and furthermore
  \[
    \int_a^b f = \int_a^c f + \int_c^b f
  .\]
\end{theorem}

\begin{proof}
    By the Archimedes-Riemann Theorem, there exists an Archimedean sequence
    \(Q_n\) of \(f\) on \([a,b]\). Define \(P_n = Q_n \cup \{c\}\) for
    every \(n \in \mb N\). This refinement is also an Archimedean sequence
    of \(f\) on \([a,b]\). Next, define \(R_n = P_n \cap [a, c]\) and
    \(S_n = P_n \cap [b, c]\) for every \(n \in \mb N\). Since 
    \(U(f, P_n) = U(f, R_n) + U(f, S_n)\) and \(L(f, P_n) = L(f, R_n) + L(f, S_n)\),
    we have
    \begin{align*}
        0 &= \limit [ U(f, P_n) - L(f, P_n) ] \\
          &= \limit [ (U(f, R_n) + U(f, S_n)) - (L(f, R_n) + L(f, S_n)) ] \\
          &= \limit [ U(f, R_n) - L(f, R_n) ] + \limit [ U(f, S_n) - L(f, S_n) ] .\\
    \end{align*}
    Since the terms within the limits are nonnegative, both limits go 
    to zero. Hence, \(R_n\) and \(S_n\) are Archimedean sequences of \(f\)
    on \([a,c]\) and \([c, b]\), respectively. Thus, \(f\) is integrable
    on \([a,c]\) and \([c, b]\). Moreover,
    \begin{align*}
        \int_a^b f &= \limit U(f, P_n) \\
                   &= \limit (U(f, R_n) + U(f, S_n)) \\
                   &= \limit U(f, R_n) + \limit U(f, S_n) \\
                   &= \int_a^c f + \int_c^b f . \\
    \end{align*}
\end{proof}


\begin{theorem}[6.13, Monotonicity of the Integral]
  Suppose \(f,g:[a,b]\to\mb R\) are integrable and that \(f(x)\leq g(x)\)
  for all \(x\in[a,b]\). Then
  \[
    \int_a^b f \leq \int_a^b g
  .\]
\end{theorem}

\begin{proof}
    By the Archimedes-Riemann theorem, we have Archimedean sequences 
    \(\{Q_n\}\) and \(\{R_n\}\)for \(f\) and \(g\) on \([a,b]\), respectively. 
    For each \(n \in \mb N\), define \(P_n = Q_n \cup R_n\). Since \(\{P_n\}\)
    refines both sequences, it is an Archimedean sequence of both \(f\) and \(g\)
    on \([a,b]\). Consequently,
    \begin{align*}
        \int_a^b f &= \limit U(f, P_n) \\
                   &\leq \limit U(g, P_n) \\
                   &= \int_a^b g \\
    \end{align*}
    following from the fact that \(f(x) \leq g(x)\) implies \(U(f, P) \leq U(g, P)\)
    (by definition) for any partition \(P\).
\end{proof}


\begin{lemma}[6.14]
  Let \(f,g:[a,b]\to\mb R\) be bounded and let \(P\) partition \([a,b]\).
  Then
  \[
    L(f,P)+L(g,P)\leq L(f+g,P)
      \text{~~and~~}
    U(f+g,P)\leq U(f,P)+U(g,P)
  .\]
  Moreover, for any number \(\alpha\),
  \[
    U(\alpha f,P)=\alpha U(f,P)
      \text{~~and~~}
    L(\alpha f,P)=\alpha L(f,P)
      \text{~~if~}
    \alpha\geq 0
  \]
  \[
    U(\alpha f,P)=\alpha L(f,P)
      \text{~~and~~}
    L(\alpha f,P)=\alpha U(f,P)
      \text{~~if~}
    \alpha< 0
  .\]
\end{lemma}

\begin{proof}
    Let \(B([a,b])\) be the set of bounded real functions on \([a,b]\).
    For partition \(P = \{x_0, x_1, ..., x_n\}\), denote \(I_i = [x_{i-1}, x_i]\) 
    for \(i \in [n]\). Then, define functions: 
    \[M_i, m_i : B([a,b]) \rightarrow \mb R \word{by}\]
    \[M_i = \sup \{ h(x) | x \in I_i \} \word{and} m_i = \inf \{ h(x) | x \in I_i \} . \]
    Then, for all \(x \in I_i\), we have
    \[ m_i(f) + m_i(g) \leq f(x) + g(x) \leq M_i(f) + M_i(g) .\]
    This shows that \(m_i(f) + m_i(g)\) is a lower bound of \(f + g\) on \(I_i\),
    and similarly for \(M_i(f) + M_i(g)\) (upper bound). Hence, by definition, and
    noting that the pointwise sum of bounded functions is bounded, we get
    \[ m_i(f) + m_i(g) \leq m_i(f + g) \word{and} M_i(f + g) \leq M_i(f) + M_i(g) .\]
    Therefore, this implies that 
    \[ L(f, P) + L(g, P) \leq L(f+g, P) \word{and} U(f + g, P) \leq U(f, P) + U(g, P) .\]

   Next, by Exercise 4, we know that 
   \[ M_i(\alpha f) = \alpha M_i(f) \word{and} m_i(\alpha f) = \alpha m_i(f) \word{if} \alpha \geq 0 \]
   and so \(U(\alpha f, P) = \alpha U(f, P)\) and \(L(\alpha f, P) = \alpha L(f, P) \).
   Likewise,
   \[ M_i(\alpha f) = \alpha m_i(f) \word{and} m_i(\alpha f) = \alpha M_i(f) \word{if} \alpha \leq 0 \]
   and thus \(U(\alpha f, P) = \alpha L(f, P)\) and \(L(\alpha f, P) = \alpha U(f, P) \).
\end{proof}


\begin{theorem}[6.15, Linearity of the Integral]
  Let \(f,g:[a,b]\to\mb R\) be integrable. Then for any two numbers
  \(\alpha,\beta\), the function \(\alpha f+\beta g:[a,b]\to\mb R\) is
  integrable and
  \[
    \int_a^b[\alpha f+\beta g]=\alpha\int_a^b f + \beta\int_a^b g
  .\]
\end{theorem}

\begin{proof}
    It suffices to show two things: first, for any integrable function 
    \(h: [a,b] \to \mb R\) and any number \(\alpha\), the function 
    \(\alpha h\) is integrable on \([a,b]\) and 
    \[ \int_a^b \alpha h = \alpha \int_a^b h .\]
    Second, for integrable functions \(f, g\) on \([a,b]\), the function
    \(f + g\) is integrable and 
    \[ \int_a^b [f + g] = \int_a^b f + \int_a^b g .\]
    From these, we get
    \begin{align*} 
        \int_a^b[ \alpha f + \beta g ] &= \int_a^b \alpha f + \int_a^b \beta g \\
                                       &= \alpha \int_a^b f + \beta \int_a^b g .\\
    \end{align*}


    We now prove the first fact. Let \(\{P_n\}\) be an Archimedean sequence
    of \(h\). Applying the previous lemma, regardless of the
    value of \(\alpha\) we have
    \[ U(\alpha h, P_n) - L(\alpha h, P_n) = |\alpha| [U(h, P_n) - L(h P_n)] \]
    and so taking the limit tells us that \(\{P_n\}\) is an Archimedean 
    sequence of \(\alpha h\). Now, for \(\alpha \geq 0\),
    \[ \int_a^b \alpha h = \limit U(\alpha h, P_n) = \alpha \limit U(h, P_n) = \alpha \int_a^b h\]
    while for \(\alpha \leq 0\),
    \[ \int_a^b \alpha h = \limit U(\alpha h, P_n) = \alpha \limit L(h, P_n) = \alpha \int_a^b h .\]
    This gives us our result.


    Now we prove the second fact. From the proof of Theorem 6.13, we know
    there exists an Archimedean sequence \(\{P_n\}\) common to both \(f\)
    and \(g\). Then, applying the previous lemma, we get
    \begin{align*}
        0 &\leq  \limit [ U(f + g, P_n) - L(f + g, P_n) ] \\ 
          &\leq \limit [ [U(f, P_n) + U(g, P_n)] - [L(f, P_n) + L(g, P_n)]] \\
          &= \limit [U(f, P_n) - L(f, P_n)] + \limit [U(g, P_n) - L(g, P_n)] \\
          &= 0 .
    \end{align*}
    This shows that \(\{P_n\}\) is an Archimedean sequence of \(f+g\). It follows
    that
    \begin{align*} 
        \int_a^b f + \int_a^b g &= \limit U(f, P_n) + \limit U(g, P_n) \\
                                &= \limit [U(f, P_n) + U(g, P_n)] \\
                                &\geq \limit [U(f + g, P_n)] \\
                                &= \int_a^b [f + g] \\
                                &= \limit [ L(f + g, P_n) ] \\
                                &\geq \limit [L(f, P_n) + L(g, P_n) ] \\
                                &= \limit L(f, P_n) + \limit L(g, P_n) \\
                                &= \int_a^b f + \int_a^b g \\
    \end{align*}
    and so \(\int_a^b [f + g] = \int_a^b f + \int_a^b g \).

\end{proof}


\begin{exercise}[1]
  Suppose that the functions \(f,g,f^2,g^2,fg\) are integrable on \([a,b]\).
  Prove that \((f-g)^2\) is also integrable on \([a,b]\) and that
  \(\int_a^b(f-g)^2\geq0\). Use this to prove that
  \[
    \int_a^b fg
      \leq
    \frac{1}{2}\left[
      \int_a^b f^2 + \int_a^b g^2
    \right]
  .\]
\end{exercise}

\begin{solution}
    Since \((f - g)^2 = f^2 - 2fg + g^2\) and each term on the RHS
    is integrable, by linearity, \((f-g)^2\) is integrable. Also,
    since \((f - g)^2 \geq 0\) (the zero function) and \(\int_a^b 0 = 0\),
    we have from the monotonicity property that
    \[ \int_a^b f^2 - 2 \int_a^b fg + \int_a^b g^2 = \int_a^b (f - g)^2 \geq \int_a^b 0 = 0 .\]
    Hence, \( \int_a^b f^2 + \int_a^b g^2 \geq 2 \int_a^b fg \) and thus
    \[ \frac{1}{2} \left[ \int_a^b f^2 + \int_a^b g^2 \right] \geq \int_a^b fg .\] 
\end{solution}

\begin{exercise}[4]
  Suppose that \(S\) is a nonempty bounded set of numbers and that \(\alpha\)
  is a number. Define \(\alpha S\) to be the set \(\{\alpha x:x\in S\}\).
  Prove that
  \[
    \sup\alpha S=\alpha\sup S
      \text{~~and~~}
    \inf\alpha S=\alpha\inf S
      \text{~~if~}
    \alpha\geq 0
  \]
  while
  \[
    \sup\alpha S=\alpha\inf S
      \text{~~and~~}
    \inf\alpha S=\alpha\sup S
      \text{~~if~}
    \alpha< 0
  .\]
\end{exercise}

\begin{solution}
    Since \(S\) is a nonempty, bounded set of real numbers, \(b = \sup S\)
    exists by the Completeness Axiom. Next, because \(\alpha\) is a fixed nonnegative
    number, \(\alpha S\) is also nonempty and bounded and so a supremum for this
    set exists. Claim: \( \sup \alpha S = \alpha b \). In the case where \(\alpha = 0\),
    then \(\alpha S = \{0\}\) and so \(\sup \alpha S = 0 = \alpha \sup S\),
    trivially. Otherwise, if \(\alpha > 0\), note that \(b \geq s \) for all \(s \in S\)
    and so \(\alpha b \geq \alpha s\). Since all elements of \(\alpha S\) take the 
    form of \(\alpha s\) for some \(s \in S\), we have that \(\alpha b\) is an upper 
    bound of \(\alpha S\). Now suppose that we have \(\alpha b > x\). 
    Dividing by \(\alpha\), we get \(b > \alpha^{-1} x \). Then, by definition,
    \(\alpha^{-1} x\) is not an upper bound of \(S\) and so there exists \(u \in S\)
    such that \(b \geq u > \alpha^{-1} x\). This implies
    \[ \alpha b \geq \alpha u > x \]
    and so \(x\) cannot be an upper bound of \(\alpha S\). Thus, \(\alpha b = \sup \alpha S\).

    Last, consider when \(\alpha < 0\) and let \(c = \inf S\). For all \(s \in S\),
    \(s \geq c\) and so multiplying by \(\alpha\) gives us \(\alpha s \leq \alpha c\). 
    By our earlier reasoning \(\alpha c\) is an upper bound of \(\alpha S\), so
    suppose \(\alpha c > x\). Divide by \(\alpha\) to get \(c < \alpha^{-1}x\).
    By definition of the infimum, there exists \(v \in S\) so that
    \(v < \alpha^{-1}x\). Multiplying by \(\alpha\) yields \(\alpha v > x \)
    and so \(x\) cannot be an upper bound of \(\alpha S\). Thus, \(\sup \alpha S = \alpha c\).

    Repeat all of the above with the set \(-S\) and note that \(\sup -S = \inf S\)
    to derive the equalities for the infimums of \(\alpha S\).
\end{solution}


\begin{exercise}[6]
  Suppose that \(f:[a,b]\to\mb R\) is bounded and let \(a<c<b\). Prove that if
  \(f\) is integrable on both \([a,c],[c,b]\), then it is integrable on
  \([a,b]\).
\end{exercise}

\begin{solution}
    By assumption, there exists Archimedean sequences \(\{Q_n\}\) and \(\{R_n\}\) on
    \([a, c]\) and \([c, d]\), respectively, such that \(\limit U(f, Q_n) = \int_a^c f\)
    and \(\limit U(f, R_n) = \int_c^b f\). For each \(n \in \mb N\), define the
    set \(P_n : Q_n \cup R_n\) (with say \(Q_n = \{q_0, ..., q_k\}\) and 
    \(R_n = \{r_0, ..., r_m\}\)). Then \(p_0 = q_0 = a\), \(p_k = q_k = c = r_0\),
    and \(p_{k+m} = r_m = b\) and since \(Q_n \cap R_n = \{c\}\), \(P_n\) is 
    a partition of \([a,b]\). Next, note that \(U(f, P_n) = U(f, Q_n) + U(f, R_n)\)
    and \(L(f, P_n) = L(f, Q_n) + L(f, R_n)\). It then follows that
    \begin{align*}
        \limit [ U(f, P_n) - L(f, P_n) ] &= \limit [ (U(f, Q_n) + U(f, R_n) - (L(f, Q_n) + L(f, R_n) ] \\
                &= \limit [U(f, Q_n) - L(f, Q_n)] + \limit [U(f, R_n) - L(f, R_n)] \\
                &= 0 + 0 \\
                &= 0 \\
    \end{align*}
    and so \(\{P_n\}\) is an Archimedean sequence. Therefore, \(f\) is integrable on \([a,b]\).
\end{solution}




\section{Continuity and Integrability}


\begin{lemma}[6.17]
  Let the function \(f:[a,b]\to\mb R\) be continuous let \(P\) partition
  its domain. Then there is a partition interval of \(P\) that contains two
  points \(u,v\) for which the following estimate holds:
  \[
    0
      \leq
    U(f,P)-L(f,P)
      \leq
    [f(v)-f(u)][b-a]
  .\]
\end{lemma}
\begin{proof}

\end{proof}


\begin{theorem}[6.18]
  A continuous function on a closed bounded interval is integrable.
\end{theorem}
\begin{proof}

\end{proof}


\begin{theorem}[6.19]
  Supose \(f:[a,b]\to\mb R\) is bounded on \([a,b]\) and continuous on
  \((a,b)\). Then \(f\) is integrable on \([a,b]\) and the value of
  \(\int_a^b f\) does not depend on the values of \(f\) at the endpoints
  of \([a,b]\).
\end{theorem}
\begin{proof}

\end{proof}

\begin{exercise}[1]
  Determine whether each of the following statements is true or false, and
  justify your answer.
  \begin{enumerate}[(a)]
    \item If \(f:[a,b]\to\mb R\) is integrable and \(\int_a^b f=0\), then
      \(f(x)=0\) for all \(x\in[a,b]\).
    \item If \(f:[a,b]\to\mb R\) is integrable, then \(f\) is continuous.
    \item If \(f:[a,b]\to\mb R\) is integrable and \(f(x)\geq0\) for all
      \(x\in[a,b]\), then \(\int_a^b f\geq 0\).
    \item A continuous function \(f:(a,b)\to\mb R\) defined on an open interval
      \((a,b)\) is bounded.
    \item A continuous function \(f:[a,b]\to\mb R\) defined on a closed interval
      \([a,b]\) is bounded.
  \end{enumerate}
\end{exercise}
\begin{solution}
  \begin{enumerate}[(a)]
    \item
    \item
    \item
    \item
    \item
  \end{enumerate}
\end{solution}


\begin{exercise}[5]
  Suppose that the continuous function \(f:[a,b]\to\mb R\) has the property
  \[
    \int_c^d f\leq 0
      \text{~~whenever~}
    a\leq c<d\leq b
  .\]
  Prove that \(f(x)\leq 0\) for all \(x\in[a,b]\). Is this true if we only
  require integrability of the function?
\end{exercise}
\begin{solution}

\end{solution}


\begin{exercise}[6]
  Suppose that \(f:[0,1]\to\mb R\) is continuous and that \(f(x)\geq 0\) for
  all \(x\in[0,1]\). Prove that \(\int_0^1 f>0\) if and only if there is a
  point \(x_0\in[0,1]\) at which \(f(x_0)>0\).
\end{exercise}
\begin{solution}

\end{solution}




\section{The First Fundamental Theorem: Integrating Derivatives}


\begin{lemma}[6.21]
  Suppose \(f:[a,b]\to\mb R\) is integrable and that the number \(A\) has
  the property that for every \(P\) partitioning \([a,b]\),
  \[
    L(f,P) \leq A \leq U(f,P)
  .\]
  Then
  \[
    \int_a^b f = A
  .\]
\end{lemma}
\begin{proof}

\end{proof}


\begin{theorem}[6.22, The First Fundamental Theorem: Integrating Derivatives]
  Let \(F:[a,b]\to\mb R\) be continuous on \([a,b]\) and differentiable on
  \((a,b)\). Moreover, suppose that its derivative
  \(F':(a,b)\to\mb R\) is both continuous and bounded. Then
  \[
    \int_a^b F'(x)~dx
      =
    F(b)-F(a)
  .\]
\end{theorem}
\begin{proof}

\end{proof}


\begin{exercise}[1]
  Let \(m,b\) be positive numbers. Find the value of \(\int_0^1 mx+b ~dx\)
  in the following three ways:
  \begin{enumerate}[(a)]
    \item Using elementary geometry, interpreting the integral as an area.
    \item Using upper and lower Darboux sums based on regular partitions of
      the interval \([0,1]\) and using the Archimedes-Riemann Theorem.
    \item Using the First Fundamental Theorem (Integrating Derivatives).
  \end{enumerate}
\end{exercise}
\begin{solution}

\end{solution}


\begin{exercise}[5]
  The monotonicity property of the integral implies that if the functions
  \(g,h:[0,\infty)\to\mb R\) are continuous and \(g(x)\leq h(x)\) for all
  \(x\geq 0\), then
  \[
    \int_0^x g\leq \int_0^x h
    \text{~~ for all~} x\geq 0
  .\]
  Use this and the First Fundamental Theorem to show that each of the following
  inequalities implies the next:
  \[
    \cos x \leq 1
    \text{~~ if~} x\geq 0
  .\]
  \[
    \sin x \leq x
    \text{~~ if~} x\geq 0
  .\]
  \[
    1-\cos x \leq \frac{x^2}{2}
    \text{~~ if~} x\geq 0
  .\]
  \[
    x-\sin x \leq \frac{x^3}{6}
    \text{~~ if~} x\geq 0
  .\]
  \[
    x-\frac{x^3}{6} \leq \sin x \leq x
    \text{~~ if~} x\geq 0
  .\]
\end{exercise}




\section{The Second Fundamental Theorem: Differentiating Integrals}


\begin{theorem}[6.26, The Mean Value Theorem for Integrals]
  Suppose that \(f:[a,b]\to\mb R\) is continuous. Then there is a point \(x_0\)
  in the interval \([a,b]\) at which
  \[
    \frac{1}{b-a}\int_a^b f
      =
    f(x_0)
  .\]
\end{theorem}
\begin{proof}

\end{proof}


\begin{proposition}[6.27]
  Suppose that the function \(f:[a,b]\to\mb R\) is integrable. Define
  \[
    F(x) = \int_a^x f
    \text{~~for all~} x\in[a,b]
  .\]
  Then the function \(F:[a,b]\to\mb R\) is continuous.
\end{proposition}
\begin{proof}

\end{proof}


\begin{theorem}[6.29, The Second Fundamental Theorem: Differentiating Integrals]
  Suppose that \(f:[a,b]\to\mb R\) is continuous. Then
  \[
    \frac{d}{dx}\left[\int_a^x\right]
      =
    f(x)
    \text{~~for all~} x\in(a,b)
  .\]
\end{theorem}
\begin{proof}

\end{proof}


\begin{exercise}[2b]
  Suppose \(f:[0,2]\to\mb R\) is defined by
  \[
    f(x) =
    \begin{cases}
      x^2 & \text{if } 0\leq x\leq 1 \\
      x   & \text{if } 1< x\leq 2
    \end{cases}
  .\]
  Define
  \[
    F(x)=\int_a^x f(t)~dt
    \text{~~for all} x\in[a,b]
  \]
  and find a formula for \(F(x)\) which does not involve integrals.
\end{exercise}
\begin{solution}

\end{solution}


\begin{exercise}[5]
  Suppose \(f:\mb R\to\mb R\) is continuous. Define
  \[
    G(x)
      =
    \int_0^x (x-t)f(t)~dt
    \text{~~for all~} x
  .\]
  Prove that \(G''(x)=f(x)\) for all \(x\).
\end{exercise}
\begin{solution}

\end{solution}


\begin{exercise}[12]
  Suppose that \(f,g:[a,b]\to\mb R\) are continuous and that \(\alpha,\beta\)
  are real numbers. Define
  \[
    H(x)
      =
    \int_a^x[\alpha f+\beta g]-\alpha\int_a^x[f]-\beta\int_a^x[g]
    \text{~~for all~} x\in[a,b]
  .\]
  Prove that \(H(a)=0\) and \(H'(x)=0\) for all \(x\in(a,b)\).
  Use this fact and the Identity Criterion to give an alternate proof of
  Theorem 6.15 for continuous functions.
\end{exercise}
\begin{solution}

\end{solution}


\setcounter{chapter}{9}
\chapter{The Euclidean Space \texorpdfstring{\(\mb R^n\)}{Rn}}
% 10.2
% 10.3
% 10.4
% 10.6
% 10.7
% ex3
% ex4
% ex9
% ex10
% 10.9
% 10.10
% ex1
% ex2
% ex5
% 10.12
% 10.13
% 10.15
% 10.16
% 10.17
% 10.19
% ex2
% ex3
% ex7
% ex12
% ex13

\end{document}
