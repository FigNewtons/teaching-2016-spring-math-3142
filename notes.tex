\documentclass[letterpaper, twoside, 12pt]{book}
\usepackage{notes}

\title{MATH 3142 Notes | Spring 2016}
\date{Updated: \today}
\author{Your Name Here\\ UNC Charlotte}

\begin{document}

\maketitle

This document is a template for you to take notes in my MATH 3142 course.
For your note check grade, you are required to complete all proofs/solutions
for the problems specified. This template will be updated periodically
throughout the course; you are responsible for updating your copy
as the template is updated. See the syllabus for more details.

You should maintain your notes on Overleaf.com and provide me with a
link so I can check on them. I'll give you notice before notes are ``due'';
when they are due I will download a copy myself from Overleaf.

This is not a replacement for the textbook for this course,
\textit{Advanced Calculus} by Patrick M. Fitzpatrick. Many proofs are
outlined in that text, as well as all the relevant definitions and other
results not included in these notes.

A proof is valid if and only if it uses concepts proven previously in
the book. For example, you cannot prove a lemma in Chapter 6 using
a theorem from Chapter 10, but using a proposition from Chapter 4
is allowed.

I hope you enjoy working through these results. Please email me with
any questions.

\noindent| Dr. Steven Clontz \(\<\)sclontz5@uncc.edu\(\>\)
















\setcounter{chapter}{5}
\chapter{Integration: Two Fundamental Theorems}







\section{Darboux Sums: Upper and Lower Integrals}

\begin{lemma}[6.1]
  Suppose that the function \(f:[a,b]\to\mb R\) is bounded and the numbers
  \(m,M\) have the property that
  \[
    m\leq f(x)\leq M
  \]
  for all \(x\) in \([a,b]\). Then, if \(P\) is a partition of the domain
  \([a,b]\),
  \[
    m(b-a)\leq L(f,P)
      \text{ and }
    U(f,P)\leq M(b-a)
  .\]
\end{lemma}
\begin{proof}

\end{proof}


\begin{lemma}[6.2, The Refinement Lemma]
  Suppose that the function \(f:[a,b]\to\mb R\) is bounded and that \(P\)
  is a partition of its domain \([a,b]\). If \(P^\star\) is a refinement
  of \(P\), then
  \[
    L(f,P)\leq L(f,P^\star)
      \text{ and }
    U(f,P^\star)\leq U(f,P)
  .\]
\end{lemma}
\begin{proof}

\end{proof}


\begin{lemma}[6.3]
  Suppose that the function \(f:[a,b]\to\mb R\) is bounded and that
  \(P_1,P_2\) are partitions of its domain. Then \(L(f,P_1)\leq U(f,P_2)\).
\end{lemma}
\begin{proof}

\end{proof}


\begin{lemma}[6.4]
  For a bounded function \(f:[a,b]\to\mb R\),
  \[
    \overline{\int_a^b} f
      \leq
    \underline{\int_a^b} f
  .\]
\end{lemma}
\begin{proof}

\end{proof}

\begin{exercise}[2]
  For an interval \([a,b]\) and a positive number \(\delta\),
  show that there is a partition \(P=\{x_i:0\leq i\leq n\}\) of
  \([a,b]\) such that each partition interval \([x_i,x_{i+1}]\)
  of \(P\) has length less than \(\delta\).
\end{exercise}
\begin{solution}

\end{solution}


\begin{exercise}[3]
  Suppose that the bounded function \(f:[a,b]\to\mb R\) has the property
  that for each rational number \(x\) in the interval \([a,b]\),
  \(f(x)=0\). Prove that
  \[
    \underline{\int_a^b}f
      \leq
    0
      \leq
    \overline{\int_a^b}f
  .\]
\end{exercise}
\begin{solution}

\end{solution}


\begin{exercise}[6]
  Suppose that \(f:[a,b]\to\mb R\) is a bounded function for which there is
  a partition \(P\) of \([a,b]\) with \(L(f,P)=U(f,P)\). Prove that
  \(f:[a,b]\to\mb R\) is constant.
\end{exercise}
\begin{solution}

\end{solution}




\section{The Archimedes-Riemann Theorem}


\begin{lemma}[6.7]
  For a bounded function \(f:[a,b]\to\mb R\) and a partition \(P\) of
  \([a,b]\),
  \[
    L(f,P)
      \leq
    \underline{\int_a^b}f\leq\overline{\int_a^b}f\leq U(f,P)
  .\]
\end{lemma}
\begin{proof}

\end{proof}


\begin{theorem}[6.8, The Archimedes-Riemann Theorem]
  Let \(f:[a,b]\to\mb R\) be a bounded function. Then \(f\) is integrable on
  \([a,b]\) if and only if there is a sequence \(\{P_n\}\) of partitions
  of the interval \([a,b]\) such that
  \[
    \lim_{n\to\infty}[U(f,P_n)-L(f,P_n)]=0
  .\]
  Moreover, for any such sequence of partitions,
  \[
    \lim_{n\to\infty} L(f,P_n)
      =
    \int_a^b f
      =
    \lim_{n\to\infty} U(f,P_n)
  .\]
\end{theorem}
\begin{proof}

\end{proof}


\begin{example}[6.9]
  Show that
  a monotonically increasing function \(f:[a,b]\to\mb R\) is integrable.
\end{example}
\begin{solution}

\end{solution}

\begin{example}[6.11]
  Show that \(\int_0^1 x^2\,dx=\frac{1}{3}\).
\end{example}
\begin{solution}

\end{solution}


\begin{exercise}[4]
  Prove that for a natural number \(n\),
  \[
    \sum_{i=1}^n i = \frac{n(n+1)}{2}
  .\]
  Then use this fact and the Archimedes-Riemann Theorem to show that
  \(\int_a^b x\,dx=(b^2-a^2)/2\).
\end{exercise}
\begin{solution}

\end{solution}


\begin{exercise}[6b]
  Use the Archimedes-Riemann Theorem to show that for \(0\leq a<b\),
  \[
    \int_a^b x^2\,dx = \frac{b^3-a^3}{3}
  .\]
\end{exercise}
\begin{solution}

\end{solution}


\begin{exercise}[9]
  Suppose that the functions \(f:[a,b]\to\mb R\) and
  \(g:[a,b]\to\mb R\) are integrable. Show that there is a sequence
  \(\{P_n\}\) of partitions of \([a,b]\) that is an Archimediean sequence
  of partitions for \(f\) on \([a,b]\) and also an Archimedean sequence
  of partitions for \(g\) on \([a,b]\).
\end{exercise}
\begin{solution}

\end{solution}


\section{Additivity, Monotonicity, and Linearity}

% 6.12
% 6.13
% 6.14
% 6.15
% ex1
% ex4
% ex6
% 6.17
% 6.18
% 6.19
% ex1
% ex5
% ex6
% 6.21
% 6.22
% ex1
% ex5
% 6.26
% 6.27
% 6.29
% ex2b
% ex5
% ex12


% 10.2
% 10.3
% 10.4
% 10.6
% 10.7
% ex3
% ex4
% ex9
% ex10
% 10.9
% 10.10
% ex1
% ex2
% ex5
% 10.12
% 10.13
% 10.15
% 10.16
% 10.17
% 10.19
% ex2
% ex3
% ex7
% ex12
% ex13

\end{document}