\documentclass[letterpaper, twoside, 12pt]{book}
\usepackage{notes}

\title{MATH 3142 Notes | Spring 2016}
\date{Updated: \today}
\author{Your Name Here\\ UNC Charlotte}

\begin{document}

\maketitle

This document is a template for you to take notes in my MATH 3142 course.
For your note check grade, you are required to complete all proofs/solutions
for the problems specified. This template will be updated periodically
throughout the course; you are responsible for updating your copy
as the template is updated. See the syllabus for more details.

You should maintain your notes on Overleaf.com and provide me with a
link so I can check on them. I'll give you notice before notes are ``due'';
when they are due I will download a copy myself from Overleaf.

This is not a replacement for the textbook for this course,
\textit{Advanced Calculus} by Patrick M. Fitzpatrick. Many proofs are
outlined in that text, as well as all the relevant definitions and other
results not included in these notes.

A proof is valid if and only if it uses concepts proven previously in
the book. For example, you cannot prove a lemma in Chapter 6 using
a theorem from Chapter 10, but using a proposition from Chapter 4
is allowed.

I hope you enjoy working through these results. Please email me with
any questions.

\noindent| Dr. Steven Clontz \(\<\)sclontz5@uncc.edu\(\>\)
















\setcounter{chapter}{5}
\chapter{Integration: Two Fundamental Theorems}







\section{Darboux Sums: Upper and Lower Integrals}

\begin{lemma}[6.1]
  Suppose that the function \(f:[a,b]\to\mb R\) is bounded and the numbers
  \(m,M\) have the property that
  \[
    m\leq f(x)\leq M
  \]
  for all \(x\) in \([a,b]\). Then, if \(P\) is a partition of the domain
  \([a,b]\),
  \[
    m(b-a)\leq L(f,P)
      \text{ and }
    U(f,P)\leq M(b-a)
  .\]
\end{lemma}
\begin{proof}

\end{proof}


\begin{lemma}[6.2, The Refinement Lemma]
  Suppose that the function \(f:[a,b]\to\mb R\) is bounded and that \(P\)
  is a partition of its domain \([a,b]\). If \(P^\star\) is a refinement
  of \(P\), then
  \[
    L(f,P)\leq L(f,P^\star)
      \text{ and }
    U(f,P^\star)\leq U(f,P)
  .\]
\end{lemma}
\begin{proof}

\end{proof}


\begin{lemma}[6.3]
  Suppose that the function \(f:[a,b]\to\mb R\) is bounded and that
  \(P_1,P_2\) are partitions of its domain. Then \(L(f,P_1)\leq U(f,P_2)\).
\end{lemma}
\begin{proof}

\end{proof}


\begin{lemma}[6.4]
  For a bounded function \(f:[a,b]\to\mb R\),
  \[
    \underline{\int_a^b} f
      \leq
    \overline{\int_a^b} f
  .\]
\end{lemma}
\begin{proof}

\end{proof}

\begin{exercise}[2]
  For an interval \([a,b]\) and a positive number \(\delta\),
  show that there is a partition \(P=\{x_i:0\leq i\leq n\}\) of
  \([a,b]\) such that each partition interval \([x_i,x_{i+1}]\)
  of \(P\) has length less than \(\delta\).
\end{exercise}
\begin{solution}

\end{solution}


\begin{exercise}[3]
  Suppose that the bounded function \(f:[a,b]\to\mb R\) has the property
  that for each rational number \(x\) in the interval \([a,b]\),
  \(f(x)=0\). Prove that
  \[
    \underline{\int_a^b}f
      \leq
    0
      \leq
    \overline{\int_a^b}f
  .\]
\end{exercise}
\begin{solution}

\end{solution}


\begin{exercise}[6]
  Suppose that \(f:[a,b]\to\mb R\) is a bounded function for which there is
  a partition \(P\) of \([a,b]\) with \(L(f,P)=U(f,P)\). Prove that
  \(f:[a,b]\to\mb R\) is constant.
\end{exercise}
\begin{solution}

\end{solution}




\section{The Archimedes-Riemann Theorem}


\begin{lemma}[6.7]
  For a bounded function \(f:[a,b]\to\mb R\) and a partition \(P\) of
  \([a,b]\),
  \[
    L(f,P)
      \leq
    \underline{\int_a^b}f\leq\overline{\int_a^b}f\leq U(f,P)
  .\]
\end{lemma}
\begin{proof}

\end{proof}


\begin{theorem}[6.8, The Archimedes-Riemann Theorem]
  Let \(f:[a,b]\to\mb R\) be a bounded function. Then \(f\) is integrable on
  \([a,b]\) if and only if there is a sequence \(\{P_n\}\) of partitions
  of the interval \([a,b]\) such that
  \[
    \lim_{n\to\infty}[U(f,P_n)-L(f,P_n)]=0
  .\]
  Moreover, for any such sequence of partitions,
  \[
    \lim_{n\to\infty} L(f,P_n)
      =
    \int_a^b f
      =
    \lim_{n\to\infty} U(f,P_n)
  .\]
\end{theorem}
\begin{proof}

\end{proof}


\begin{example}[6.9]
  Show that
  a monotonically increasing function \(f:[a,b]\to\mb R\) is integrable.
\end{example}
\begin{solution}

\end{solution}

\begin{example}[6.11]
  Show that \(\int_0^1 x^2\,dx=\frac{1}{3}\).
\end{example}
\begin{solution}

\end{solution}


\begin{exercise}[4]
  Prove that for a natural number \(n\),
  \[
    \sum_{i=1}^n i = \frac{n(n+1)}{2}
  .\]
  Then use this fact and the Archimedes-Riemann Theorem to show that
  \(\int_a^b x\,dx=(b^2-a^2)/2\).
\end{exercise}
\begin{solution}

\end{solution}


\begin{exercise}[6b]
  Use the Archimedes-Riemann Theorem to show that for \(0\leq a<b\),
  \[
    \int_a^b x^2\,dx = \frac{b^3-a^3}{3}
  .\]
\end{exercise}
\begin{solution}

\end{solution}


\begin{exercise}[9]
  Suppose that the functions \(f:[a,b]\to\mb R\) and
  \(g:[a,b]\to\mb R\) are integrable. Show that there is a sequence
  \(\{P_n\}\) of partitions of \([a,b]\) that is an Archimediean sequence
  of partitions for \(f\) on \([a,b]\) and also an Archimedean sequence
  of partitions for \(g\) on \([a,b]\).
\end{exercise}
\begin{solution}

\end{solution}




\section{Additivity, Monotonicity, and Linearity}


\begin{theorem}[6.12, Additivity over Intervals]
  Let \(f:[a,b]\to\mb R\) be integrable on \([a,b]\) and let \(c\in(a,b)\).
  Then \(f\) is integrable on \([a,c]\) and \([c,b]\), and furthermore
  \[
    \int_a^b f = \int_a^c f + \int_c^b f
  .\]
\end{theorem}
\begin{proof}

\end{proof}


\begin{theorem}[6.13, Monotonicity of the Integral]
  Suppose \(f,g:[a,b]\to\mb R\) are integrable and that \(f(x)\leq g(x)\)
  for all \(x\in[a,b]\). Then
  \[
    \int_a^b f \leq \int_a^b g
  .\]
\end{theorem}
\begin{proof}

\end{proof}


\begin{lemma}[6.14]
  Let \(f,g:[a,b]\to\mb R\) be bounded and let \(P\) partition \([a,b]\).
  Then
  \[
    L(f,P)+L(g,P)\leq L(f+g,P)
      \text{~~and~~}
    U(f+g,P)\leq U(f,P)+U(g,P)
  .\]
  Moreover, for any number \(\alpha\),
  \[
    U(\alpha f,P)=\alpha U(f,P)
      \text{~~and~~}
    L(\alpha f,P)=\alpha L(f,P)
      \text{~~if~}
    \alpha\geq 0
  \]
  \[
    U(\alpha f,P)=\alpha L(f,P)
      \text{~~and~~}
    L(\alpha f,P)=\alpha U(f,P)
      \text{~~if~}
    \alpha< 0
  .\]
\end{lemma}
\begin{proof}

\end{proof}


\begin{theorem}[6.15, Linearity of the Integral]
  Let \(f,g:[a,b]\to\mb R\) be integrable. Then for any two numbers
  \(\alpha,\beta\), the function \(\alpha f+\beta g:[a,b]\to\mb R\) is
  integrable and
  \[
    \int_a^b[\alpha f+\beta g]=\alpha\int_a^b f + \beta\int_a^b g
  .\]
\end{theorem}
\begin{proof}

\end{proof}


\begin{exercise}[1]
  Suppose that the functions \(f,g,f^2,g^2,fg\) are integrable on \([a,b]\).
  Prove that \((f-g)^2\) is also integrable on \([a,b]\) and that
  \(\int_a^b(f-g)^2\geq0\). Use this to prove that
  \[
    \int_a^b fg
      \leq
    \frac{1}{2}\left[
      \int_a^b f^2 + \int_a^b g^2
    \right]
  .\]
\end{exercise}
\begin{solution}

\end{solution}

\begin{exercise}[4]
  Suppose that \(S\) is a nonempty bounded set of numbers and that \(\alpha\)
  is a number. Define \(\alpha S\) to be the set \(\{\alpha x:x\in S\}\).
  Prove that
  \[
    \sup\alpha S=\alpha\sup S
      \text{~~and~~}
    \inf\alpha S=\alpha\inf S
      \text{~~if~}
    \alpha\geq 0
  \]
  while
  \[
    \sup\alpha S=\alpha\inf S
      \text{~~and~~}
    \inf\alpha S=\alpha\sup S
      \text{~~if~}
    \alpha< 0
  .\]
\end{exercise}
\begin{solution}

\end{solution}


\begin{exercise}[6]
  Suppose that \(f:[a,b]\to\mb R\) is bounded and let \(a<c<b\). Prove that if
  \(f\) is integrable on both \([a,c],[c,b]\), then it is integrable on
  \([a,b]\).
\end{exercise}
\begin{solution}

\end{solution}




\section{Continuity and Integrability}


\begin{lemma}[6.17]
  Let the function \(f:[a,b]\to\mb R\) be continuous let \(P\) partition
  its domain. Then there is a partition interval of \(P\) that contains two
  points \(u,v\) for which the following estimate holds:
  \[
    0
      \leq
    U(f,P)-L(f,P)
      \leq
    [f(v)-f(u)][b-a]
  .\]
\end{lemma}
\begin{proof}

\end{proof}


\begin{theorem}[6.18]
  A continuous function on a closed bounded interval is integrable.
\end{theorem}
\begin{proof}

\end{proof}


\begin{theorem}[6.19]
  Supose \(f:[a,b]\to\mb R\) is bounded on \([a,b]\) and continuous on
  \((a,b)\). Then \(f\) is integrable on \([a,b]\) and the value of
  \(\int_a^b f\) does not depend on the values of \(f\) at the endpoints
  of \([a,b]\).
\end{theorem}
\begin{proof}

\end{proof}

\begin{exercise}[1]
  Determine whether each of the following statements is true or false, and
  justify your answer.
  \begin{enumerate}[(a)]
    \item If \(f:[a,b]\to\mb R\) is integrable and \(\int_a^b f=0\), then
      \(f(x)=0\) for all \(x\in[a,b]\).
    \item If \(f:[a,b]\to\mb R\) is integrable, then \(f\) is continuous.
    \item If \(f:[a,b]\to\mb R\) is integrable and \(f(x)\geq0\) for all
      \(x\in[a,b]\), then \(\int_a^b f\geq 0\).
    \item A continuous function \(f:(a,b)\to\mb R\) defined on an open interval
      \((a,b)\) is bounded.
    \item A continuous function \(f:[a,b]\to\mb R\) defined on a closed interval
      \([a,b]\) is bounded.
  \end{enumerate}
\end{exercise}
\begin{solution}
  \begin{enumerate}[(a)]
    \item
    \item
    \item
    \item
    \item
  \end{enumerate}
\end{solution}


\begin{exercise}[5]
  Suppose that the continuous function \(f:[a,b]\to\mb R\) has the property
  \[
    \int_c^d f\leq 0
      \text{~~whenever~}
    a\leq c<d\leq b
  .\]
  Prove that \(f(x)\leq 0\) for all \(x\in[a,b]\). Is this true if we only
  require integrability of the function?
\end{exercise}
\begin{solution}

\end{solution}


\begin{exercise}[6]
  Suppose that \(f:[0,1]\to\mb R\) is continuous and that \(f(x)\geq 0\) for
  all \(x\in[0,1]\). Prove that \(\int_0^1 f>0\) if and only if there is a
  point \(x_0\in[0,1]\) at which \(f(x_0)>0\).
\end{exercise}
\begin{solution}

\end{solution}




\section{The First Fundamental Theorem: Integrating Derivatives}


\begin{lemma}[6.21]
  Suppose \(f:[a,b]\to\mb R\) is integrable and that the number \(A\) has
  the property that for every \(P\) partitioning \([a,b]\),
  \[
    L(f,P) \leq A \leq U(f,P)
  .\]
  Then
  \[
    \int_a^b f = A
  .\]
\end{lemma}
\begin{proof}

\end{proof}


\begin{theorem}[6.22, The First Fundamental Theorem: Integrating Derivatives]
  Let \(F:[a,b]\to\mb R\) be continuous on \([a,b]\) and differentiable on
  \((a,b)\). Moreover, suppose that its derivative
  \(F':(a,b)\to\mb R\) is both continuous and bounded. Then
  \[
    \int_a^b F'(x)~dx
      =
    F(b)-F(a)
  .\]
\end{theorem}
\begin{proof}

\end{proof}


\begin{exercise}[1]
  Let \(m,b\) be positive numbers. Find the value of \(\int_0^1 mx+b ~dx\)
  in the following three ways:
  \begin{enumerate}[(a)]
    \item Using elementary geometry, interpreting the integral as an area.
    \item Using upper and lower Darboux sums based on regular partitions of
      the interval \([0,1]\) and using the Archimedes-Riemann Theorem.
    \item Using the First Fundamental Theorem (Integrating Derivatives).
  \end{enumerate}
\end{exercise}
\begin{solution}

\end{solution}


\begin{exercise}[5]
  The monotonicity property of the integral implies that if the functions
  \(g,h:[0,\infty)\to\mb R\) are continuous and \(g(x)\leq h(x)\) for all
  \(x\geq 0\), then
  \[
    \int_0^x g\leq \int_0^x h
    \text{~~ for all~} x\geq 0
  .\]
  Use this and the First Fundamental Theorem to show that each of the following
  inequalities implies the next:
  \[
    \cos x \leq 1
    \text{~~ if~} x\geq 0
  .\]
  \[
    \sin x \leq x
    \text{~~ if~} x\geq 0
  .\]
  \[
    1-\cos x \leq \frac{x^2}{2}
    \text{~~ if~} x\geq 0
  .\]
  \[
    x-\sin x \leq \frac{x^3}{6}
    \text{~~ if~} x\geq 0
  .\]
  \[
    x-\frac{x^3}{6} \leq \sin x \leq x
    \text{~~ if~} x\geq 0
  .\]

  (For this problem, you may assume that the sine and cosine functions
  are differentiable functions with the properties
  \(\sin(0)=0\), \(\cos(0)=1\), \(\frac{d}{dx}[\sin(x)]=\cos(x)\),
  and \(\frac{d}{dx}[\cos(x)]=-\sin(x)\).)
\end{exercise}
\begin{solution}

\end{solution}




\section{The Second Fundamental Theorem: Differentiating Integrals}


\begin{theorem}[6.26, The Mean Value Theorem for Integrals]
  Suppose that \(f:[a,b]\to\mb R\) is continuous. Then there is a point \(x_0\)
  in the interval \([a,b]\) at which
  \[
    \frac{1}{b-a}\int_a^b f
      =
    f(x_0)
  .\]
\end{theorem}
\begin{proof}

\end{proof}


\begin{proposition}[6.27]
  Suppose that the function \(f:[a,b]\to\mb R\) is integrable. Define
  \[
    F(x) = \int_a^x f
    \text{~~for all~} x\in[a,b]
  .\]
  Then the function \(F:[a,b]\to\mb R\) is continuous.
\end{proposition}
\begin{proof}

\end{proof}


\begin{theorem}[6.29, The Second Fundamental Theorem: Differentiating Integrals]
  Suppose that \(f:[a,b]\to\mb R\) is continuous. Then
  \[
    \frac{d}{dx}\left[\int_a^x\right]
      =
    f(x)
    \text{~~for all~} x\in(a,b)
  .\]
\end{theorem}
\begin{proof}

\end{proof}


\begin{exercise}[2b]
  Suppose \(f:[0,2]\to\mb R\) is defined by
  \[
    f(x) =
    \begin{cases}
      x^2 & \text{if } 0\leq x\leq 1 \\
      x   & \text{if } 1< x\leq 2
    \end{cases}
  .\]
  Define
  \[
    F(x)=\int_a^x f(t)~dt
    \text{~~for all} x\in[a,b]
  \]
  and find a formula for \(F(x)\) which does not involve integrals.
\end{exercise}
\begin{solution}

\end{solution}


\begin{exercise}[5]
  Suppose \(f:\mb R\to\mb R\) is continuous. Define
  \[
    G(x)
      =
    \int_0^x (x-t)f(t)~dt
    \text{~~for all~} x
  .\]
  Prove that \(G''(x)=f(x)\) for all \(x\).
\end{exercise}
\begin{solution}

\end{solution}


\begin{exercise}[12]
  Suppose that \(f,g:[a,b]\to\mb R\) are continuous and that \(\alpha,\beta\)
  are real numbers. Define
  \[
    H(x)
      =
    \int_a^x[\alpha f+\beta g]-\alpha\int_a^x[f]-\beta\int_a^x[g]
    \text{~~for all~} x\in[a,b]
  .\]
  Prove that \(H(a)=0\) and \(H'(x)=0\) for all \(x\in(a,b)\).
  Use this fact and the Identity Criterion to give an alternate proof of
  Theorem 6.15 for continuous functions.
\end{exercise}
\begin{solution}

\end{solution}



\setcounter{chapter}{9}
\chapter{The Euclidean Space \texorpdfstring{$\mb R^n$}{Rn}}


\section{The Linear Structure of \texorpdfstring{$\mb R^n$}{Rn}
and the Scalar Product}

\begin{proposition}[10.2]
  Let \(\vect u,\vect v,\vect w\in\mb R^n\)
  and \(\alpha,\beta\in\mb R\). Then both of the following hold:
  \[
    \<\vect u,\vect v\>=\<\vect v,\vect u\>
  \]
  \[
    \<\alpha\vect u+\beta\vect w,v\>
      =
    \alpha\<\vect u,\vect v\>+\beta\<\vect w,\vect v\>
  \]
\end{proposition}
\begin{proof}

\end{proof}

\begin{lemma}[10.4]
  For \(\vect u,\vect v\in\mb R^n\), \(\vect u,\vect v\) are
  orthogonal if and only if
  \(\|\vect u+\vect v\|=\|\vect u\|^2+\|\vect v\|^2\).
\end{lemma}
\begin{proof}

\end{proof}

\begin{lemma}[10.5]
  For \(\vect u,\vect v\in\mb R^n\) where \(\vect v\not=\vect 0\),
  define \(\lambda=\frac{\<\vect u,\vect v\>}{\<\vect v,\vect v\>}\)
  and \(\vect w=\vect u-\lambda\vect v\). Then \(\vect v,\vect w\)
  are orthogonal and \(\vect u=\vect w+\lambda\vect v\).
\end{lemma}
\begin{proof}

\end{proof}

\begin{theorem}[10.6, The Cauchy-Schwarz Inequality]
  For any two vectors \(\vect u,\vect v\in\mb R^n\),
  \[
    |\<\vect u,\vect v\>|
      \leq
    \|\vect u\|\|\vect v\|
  .\]
\end{theorem}
\begin{proof}

\end{proof}

\begin{theorem}[10.7, The Triangle Inequality]
  For any two vectors \(\vect u,\vect v\in\mb R^n\),
  \[
    \|\vect u+\vect v\|
      \leq
    \|\vect u\|+\|\vect v\|
  .\]
\end{theorem}

\begin{exercise}[3]
  Show that for \(\vect u\in\mb R^n\), \(\alpha\in\mb R\):
  \begin{enumerate}[(a)]
    \item \(\|\vect u\|=0\) if and only if \(\vect u=\vect 0\).
    \item \(\|\alpha\vect u\|=|\alpha|\|\vect u\|\).
  \end{enumerate}
\end{exercise}
\begin{proof}

\end{proof}

\begin{exercise}[4]
  For \(\vect u,\vect v\in\mathbb R^n\) verify the identity
  \[
    \|\vect u-\vect v\|^2
      =
    \|\vect u\|^2+\|\vect v\|^2-2\<\vect u,\vect v\>
  .\]
\end{exercise}
\begin{solution}

\end{solution}

\begin{exercise}[9]
  Let \(\vect u\in\mb R^n\) and suppose \(\|\vect u\|<1\).
  Show that for \(\vect v\in\mb R^n\),
  \(\|\vect v-\vect u\|<1-\|\vect u\|\) implies
  \(\|\vect v\|<1\).
\end{exercise}
\begin{solution}

\end{solution}

\begin{exercise}[10]
  Let \(\vect u\in\mb R^n\) and \(r>0\). Suppose \(\vect v,\vect w\in\mb R^n\)
  are at a distance less than \(r\) from \(\vect u\). Prove that if
  \(0\leq t\leq 1\), then the point \(t\vect v+(1-t)]\vect w\) is also
  at a distance less than \(r\) from \(\vect u\).
\end{exercise}
\begin{solution}

\end{solution}


\section{Convergence of Sequences in \texorpdfstring{$\mb R^n$}{Rn}}

\begin{theorem}[10.9, The Componentwise Convergence Criterion]
  Let \(\{\vect u_k\}\) be a sequence in \(\mb R^n\). Then
  \(\{\vect u_k\}\) converges to \(\vect u\) if and only if
  \(\{p_i(\vect u_k)\}\) converges to \(p_i(\vect u)\) for
  all \(1\leq i\leq n\).
\end{theorem}
\begin{proof}

\end{proof}

\begin{theorem}[10.10]
  Let \(\{\vect u_k\}\), \(\{\vect v_k\}\) be sequences in \(\mb R^n\)
  such that \(\{\vect u_k\}\) converges to \(\vect u\) and
  \(\{\vect v_k\}\) converges to \(\vect v\). Then for any
  \(\alpha,\beta\in\mb R\),
  \[
    \lim_{k\to\infty}[\alpha\vect u_k+\beta\vect v_k]
      =
    \alpha\vect u+\beta\vect v
  .\]
\end{theorem}
\begin{proof}

\end{proof}

\begin{exercise}[1]
  Let \(\{\vect u_k\}\) be a sequence in \(\mb R^n\) that converges to
  \(\vect u\). Prove the following for all \(\vect v\in\mb R^n\):
  \[
    \lim_{k\to\infty}\<\vect u_k,\vect v\>=\<\vect u,\vect v\>
  .\]
\end{exercise}
\begin{solution}

\end{solution}

\begin{exercise}[2]
  Let \(\{\vect u_k\}\) be a sequence in \(\mb R^n\) and
  \(\vect u\in\mb R^n\). Prove that if
  \[
    \lim_{k\to\infty}\<\vect u_k,\vect v\>=\<\vect u,\vect v\>
  \]
  holds for all \(\vect v\in\mb R^n\), then \(\{\vect u_k\}\) converges
  to \(\vect u\).
\end{exercise}
\begin{solution}

\end{solution}

\begin{exercise}[5]
  Let \(\{\vect u_k\}\) be a sequence in \(\mb R^n\) that converges to
  \(\vect u\) where \(\|\vect u\|=>0\). Prove that there is an index \(K\)
  where
  \[
    \|\vect u_k\|>\frac{r}{2}
    \text{~~if~}
    k\geq K
  .\]
\end{exercise}
\begin{solution}

\end{solution}


\section{Open Sets and Closed Sets in \texorpdfstring{$\mb R^n$}{Rn}}


\begin{example}[10.11]
  Let \(a<b\) be in \(\mb R\). Then \(\interior(a,b]=(a,b)\).
\end{example}
\begin{proof}

\end{proof}

\begin{example}[10.12]
  Let \(\mb Q\subseteq\mb R\) be the set of rational real numbers.
  Then \(\interior\mb Q=\emptyset\).
\end{example}
\begin{proof}

\end{proof}

\begin{proposition}[10.13]
  Every open ball \(B_r(\vect u)\) in \(\mb R^n\) is open.
\end{proposition}
\begin{proof}

\end{proof}

\begin{example}[10.14]
  Let \(a<b\) be in \(\mb R\). Then \([a,b]\) is closed.
\end{example}
\begin{proof}

\end{proof}

\begin{example}[10.15]
  The set
  \[
    [-1,1]\times[-1,1]
      =
    \{(x,y)\in\mb R^2:-1\leq x\leq 1 \text{~and~} -1\leq y\leq 1\}
  \]
  is closed in \(\mb R^2\).
\end{example}
\begin{proof}

\end{proof}

\begin{theorem}[10.16, The Complementing Characterization]
  A subset \(A\subseteq\mb R^n\) is open if and only if its complement
  \(\mb R^n\setminus A\) is closed.
\end{theorem}
\begin{proof}

\end{proof}

\begin{proposition}[10.17.i]
  The union of a collection of open subsets of \(\mb R^n\) is open.
\end{proposition}
\begin{proof}

\end{proof}

\begin{proposition}[10.17.ii]
  The intersection of a collection of closed subsets of \(\mb R^n\) is closed.
\end{proposition}
\begin{proof}

\end{proof}

\begin{proposition}[10.18.i]
  The intersection of a finite collection of
  open subsets of \(\mb R^n\) is open.
\end{proposition}
\begin{proof}

\end{proof}

\begin{proposition}[10.18.ii]
  The union of a finite collection of
  closed subsets of \(\mb R^n\) is closed.
\end{proposition}
\begin{proof}

\end{proof}

\begin{proposition}[10.19.i]
  \(A\subseteq\mb R^n\) is open if and only if \(A\cap\boundary A=\emptyset\).
\end{proposition}
\begin{proof}

\end{proof}

\begin{proposition}[10.19.ii]
  \(A\subseteq\mb R^n\) is closed if and only if \(\boundary A\subseteq A\).
\end{proposition}
\begin{proof}

\end{proof}

\begin{exercise}[2]
  Determine which of the following subsets of \(\mb R^2\) are open, closed,
  neither, or both.
  \begin{enumerate}[(a)]
    \item \(\{(x,y):x^2>y\}\)
    \item \(\{(x,y):x^2+y^2=1\}\)
    \item \(\{(x,y):x \text{~is rational}\}\)
    \item \(\{(x,y):x\geq 0,~y\geq0\}\)
  \end{enumerate}
\end{exercise}
\begin{solution}
  \begin{enumerate}[(a)]
    \item
    \item
    \item
    \item
  \end{enumerate}
\end{solution}

\begin{exercise}[3]
  Let \(r>0\) and \(O=\{\vect u\in\mb R^n:\|\vect u\|>r\}\). Prove that \(O\)
  is open.
\end{exercise}
\begin{solution}

\end{solution}

\begin{exercise}[7a]
  Show that \(A\subseteq\mb R^n\) is open if and only if
  \[
    \vect w + A = \{\vect w+\vect u:\vect u\in A\}
  \]
  is open for all \(\vect w\in\mb R^n\).
\end{exercise}
\begin{solution}

\end{solution}

\begin{exercise}[12]
  For \(A\subseteq\mb R^n\), denote its closure by
  \[
    \closure A = \interior A \cup \boundary A
  .\]
  Prove that \(A\subseteq \closure A\). Then prove that
  \(A=\closure A\) if and only if \(A\) is closed.
\end{exercise}
\begin{solution}

\end{solution}





\chapter{Continuity, Compactness, and Connectedness}

\section{Continuous Functions and Mappings}


\end{document}


