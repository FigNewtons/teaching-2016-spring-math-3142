\documentclass[letterpaper, twoside, 12pt]{book}
\usepackage{notes}




\title{MATH 3142 Notes | Spring 2016}
\date{Updated: \today}
\author{Daniel Gruszczynski\\ UNC Charlotte}

\begin{document}

\maketitle

This document is a template for you to take notes in my MATH 3142 course.
For your note check grade, you are required to complete all proofs/solutions
for the problems specified. This template will be updated periodically
throughout the course; you are responsible for updating your copy
as the template is updated. See the syllabus for more details.

You should maintain your notes on Overleaf.com and provide me with a
link so I can check on them. I'll give you notice before notes are ``due'';
when they are due I will download a copy myself from Overleaf.

This is not a replacement for the textbook for this course,
\textit{Advanced Calculus} by Patrick M. Fitzpatrick. Many proofs are
outlined in that text, as well as all the relevant definitions and other
results not included in these notes.

A proof is valid if and only if it uses concepts proven previously in
the book. For example, you cannot prove a lemma in Chapter 6 using
a theorem from Chapter 10, but using a proposition from Chapter 4
is allowed.

I hope you enjoy working through these results. Please email me with
any questions.

\noindent| Dr. Steven Clontz \(\<\)sclontz5@uncc.edu\(\>\)
















\setcounter{chapter}{5}
\chapter{Integration: Two Fundamental Theorems}







\section{Darboux Sums: Upper and Lower Integrals}

\begin{definition}
    Let \(n\) be a natural number. We define \([n] = \{1, 2, ..., n\}\). If 
    \(i\) is an index, we write ``for \(i \in [n]\)'' in place of the usual 
    ``for \(1 \leq i \leq n\).''
\end{definition}

\begin{lemma}[6.1]
  Suppose that the function \(f:[a,b]\to\mb R\) is bounded and the numbers
  \(m,M\) have the property that
  \[
    m\leq f(x)\leq M
  \]
  for all \(x\) in \([a,b]\). Then, if \(P\) is a partition of the domain
  \([a,b]\),
  \[
    m(b-a)\leq L(f,P)
      \text{ and }
    U(f,P)\leq M(b-a)
  .\]
\end{lemma}

\begin{proof}
    Let \(P = \{x_{0}, x_{1}, ..., x_{n}\}\) be a partition on \([a,b]\).
    By assumption, \(m\) is a lower bound of \(f([a,b])\). Restricting
    \(f\) to \([x_{i-1}, x_{i}]\), we have \(m \leq m_{i}\) for all \(i \in [n]\)
    since \(m_{i}\) is the infimum of \(f([x_{i-1}, x_{i}])\). Then, by definition,
    \begin{align*}
        L(f, P) &= \sum_{i = 1}^{n} m_{i}(x_{i} - x_{i-1}) \\ 
                &\geq \sum_{i = 1}^{n} m (x_{i} - x_{i-1}) \\
                &= m \sum_{i = 1}^{n} (x_{i} - x_{i-1}) \\
                &= m (b - a) .\\
    \end{align*}
    Similarly, \(M\) is an upper bound of \(f([a,b])\) and so, when restricting \(f\) 
    to \([x_{i-1}, x_{i}]\), we have \(M \geq M_{i}\) since \(M_{i}\) is the supremum of 
    \(f([x_{i-1}, x_{i}])\) (for all \(i \in [n]\)). Hence, we have
    \begin{align*}
        U(f, P) &= \sum_{i = 1}^{n} M_{i}(x_{i} - x_{i - 1}) \\
                &\leq \sum_{i = 1}^{n} M(x_{i} - x_{i - 1}) \\
                &= M \sum_{i = 1}^{n} (x_{i} - x_{i - 1}) \\
                &= M(b - a) .\\
    \end{align*}

    Therefore, \(m(b - a) \leq L(f, P)\) and \(U(f, P) \leq M(b - a) \). 
\end{proof}


\begin{lemma}[6.2, The Refinement Lemma]
  Suppose that the function \(f:[a,b]\to\mb R\) is bounded and that \(P\)
  is a partition of its domain \([a,b]\). If \(P^\star\) is a refinement
  of \(P\), then
  \[
    L(f,P)\leq L(f,P^\star)
      \text{ and }
    U(f,P^\star)\leq U(f,P)
  .\]
\end{lemma}

\begin{proof}
    Let \(P = \{x_{0}, x_{1}, ..., x_{n}\}\) be a partition on \([a,b]\),
    and let \(P^*\) be its refinement. For \(i \in [n]\), define \(P_{i}\) 
    to be the partition on \([x_{i-1}, x_{i}]\) by the points of 
    \(P^*\) inside this interval. Since \(m_{i} \leq f(x)\) for \(x \in [x_{i-1},x_{i}]\),
    applying the previous lemma to the restriction of \(f\) on \([x_{i-1}, x_{i}]\),
    we have \( m_{i}(x_{i} - x_{i-1}) \leq L(f, P_{i})\). It follows that
    \begin{align*} 
        L(f, P) &= \sum_{i=1}^{n} m_{i}(x_{i} - x_{i - 1}) \\
                &\leq \sum_{i = 1}^{n} L(f, P_{i}) \\
                &= L(f, P^*). \\
    \end{align*}

    Likewise, the previous lemma gives us \(M_{i}(x_{i} - x_{i-1}) \geq U(f, P_{i})\).
    Hence,
    \begin{align*}
        U(f, P) &= \sum_{i=1}^{n} M_{i}(x_{i} - x_{i-1}) \\
                &\geq \sum_{i=1}^{n} U(f, P_{i}) \\
                &= U(f, P^*). \\
    \end{align*}
\end{proof}


\begin{lemma}[6.3]
  Suppose that the function \(f:[a,b]\to\mb R\) is bounded and that
  \(P_1,P_2\) are partitions of its domain. Then \(L(f,P_1)\leq U(f,P_2)\).
\end{lemma}

\begin{proof}
    Let \(P = P_1 \cup P_2\) be the common refinement of partitions \(P_1\)
    and \(P_2\). By the Refinement Lemma, \(L(f, P_{1}) \leq L(f, P)\)
    and \(U(f, P) \leq U(f, P_{2})\). Then, since \(L(f, P) \leq U(f, P)\),
    the transitivity of \(\leq\) implies that \(L(f, P_{1}) \leq U(f, P_{2})\).
\end{proof}


\begin{lemma}[6.4]
  For a bounded function \(f:[a,b]\to\mb R\),
  \[
    \underline{\int_a^b} f
      \leq
    \overline{\int_a^b} f
  .\]
\end{lemma}

\begin{proof}
    Let \(P\) be any partition on \([a,b]\). By the previous lemma,
    \(U(f, P) \geq L(f, P')\) for all partitions \(P'\) on \([a,b]\). 
    It follows that
    \[ \underline{\int_a^b} f \leq U(f, P) . \]
    Since \(P\) was arbitrary, the above shows that \(\underline{\int_a^b} f\)
    is a lower bound for all such \(U(f, P)\). Therefore,
    \[ \underline{\int_a^b} f \leq \overline{\int_a^b} f .\]
\end{proof}

\begin{exercise}[2]
  For an interval \([a,b]\) and a positive number \(\delta\),
  show that there is a partition \(P=\{x_i:0\leq i\leq n\}\) of
  \([a,b]\) such that each partition interval \([x_i,x_{i+1}]\)
  of \(P\) has length less than \(\delta\).
\end{exercise}

\begin{solution}
    Let \([a,b]\) be an interval (\(b > a\)) and \(\delta > 0\).
    By the Archimedean property, there exists a natural number
    \(n\) such that \(\frac{\delta}{b - a} > \frac{1}{n}\). It follows that
    we can form partition intervals of equal length \(\frac{b - a}{n}\):
    \begin{align*}
        \delta &> \frac{b - a}{n} \\
               &= \frac{1}{n} \sum_{i=0}^{n - 1} (x_{i + 1} - x_{i}) \\
               &= \frac{1}{n}[ n (x_{i + 1} - x_{i}) ] \\
               &= x_{i + 1} - x_{i} \\
    \end{align*}
\end{solution}


\begin{exercise}[3]
  Suppose that the bounded function \(f:[a,b]\to\mb R\) has the property
  that for each rational number \(x\) in the interval \([a,b]\),
  \(f(x)=0\). Prove that
  \[
    \underline{\int_a^b}f
      \leq
    0
      \leq
    \overline{\int_a^b}f
  .\]
\end{exercise}

\begin{solution}
    Let \(P = \{x_{0}, x_{1}, ... , x_{n}\}\) be an arbitrary partition on \([a,b]\).
    Since \(\mb Q\) is dense in \(\mb R\), \(m_{i} \leq 0\) and \(M_{i} \geq 0\) for
    all \(i \in [n]\). This implies \(L(f, P) \leq 0\) and \(U(f, P) \geq 0\). 
    Consequently, 
    \[ \underline{\int_a^b} f \leq 0 \leq \overline{\int_a^b} f .\]
\end{solution}


\begin{exercise}[6]
  Suppose that \(f:[a,b]\to\mb R\) is a bounded function for which there is
  a partition \(P\) of \([a,b]\) with \(L(f,P)=U(f,P)\). Prove that
  \(f:[a,b]\to\mb R\) is constant.
\end{exercise}

\begin{solution}
    Let \(P\) be the partition where \(L(f, P) = U(f, P)\). Then
    \[ 0 = U(f, P) - L(f, P) = \sum_{i=1}^{n} M_i(x_{i} - x_{i-1}) -
                                \sum_{i = 1}^{n} m_i (x_{i} - x_{i-1}) 
                            = \sum_{i=1}^{n} (M_{i} - m_{i})(x_{i} - x_{i-1}) .\]
    Since \(x_{i} > x_{i-1}\), \((x_{i} - x_{i-1}) > 0\).  Similarly,
    \((M_{i} - m_{i}) \geq 0\). This implies that the term 
    \((M_{i} - m_{i})(x_{i} - x_{i-1})\) is nonnegative, but since 
    the entire sum is zero, we must have \(M_{i} = m_{i}\) for
    all \(i \in [n]\). It follows that \(f\) takes the same value
    within each partition interval, and since 
    \([x_{i - 1}, x_{i}] \cap [x_{i}, x_{i + 1}] = \{x_{i}\}\), \(f\)
    takes the same value for all of \([a,b]\). Therefore, \(f\)
    is constant.
\end{solution}




\section{The Archimedes-Riemann Theorem}


\begin{lemma}[6.7]
  For a bounded function \(f:[a,b]\to\mb R\) and a partition \(P\) of
  \([a,b]\),
  \[
    L(f,P)
      \leq
    \underline{\int_a^b}f\leq\overline{\int_a^b}f\leq U(f,P)
  .\]
\end{lemma}

\begin{proof}
    By definition, \(\overline{\int_a^b} f \leq U(f, P)\) and
    \(L(f, P) \leq \underline{\int_a^b} f\). Then, by Lemma 6.4
    we have \(\underline{\int_a^b} f \leq \overline{\int_a^b} f\).
    The result follows.
\end{proof}


\begin{theorem}[6.8, The Archimedes-Riemann Theorem]
  Let \(f:[a,b]\to\mb R\) be a bounded function. Then \(f\) is integrable on
  \([a,b]\) if and only if there is a sequence \(\{P_n\}\) of partitions
  of the interval \([a,b]\) such that
  \[
    \lim_{n\to\infty}[U(f,P_n)-L(f,P_n)]=0
  .\]
  Moreover, for any such sequence of partitions,
  \[
    \lim_{n\to\infty} L(f,P_n)
      =
    \int_a^b f
      =
    \lim_{n\to\infty} U(f,P_n)
  .\]
\end{theorem}

\begin{proof}
    Suppose \(f\) is integrable on \([a,b]\). Then by definition
    \[ \underline{\int_a^b} f = \int_a^b f = \overline{\int_a^b} f .\]
    For convenience, let \(L = \underline{\int_a^b} f\) and
    \(U = \overline{\int_a^b} f\). Now, for each \(n \in \mb N\), 
    define \(L_n \equiv L - \frac{1}{n}\) and \(U_n \equiv U + \frac{1}{n}\).
    Since \(L\) is the supremum of the lower Darboux sums of \(f\),
    \(L_n\) is not an upper bound of this collection and so there
    exists a partition \(P'\) such that \(L_n < L(f, P')\). By similar
    reasoning, there exists a partition \(P''\) such that
    \(U(f, P'') < U_n \). Define \(P_n = P' \cup P''\) as their common
    refinement. This gives us
    \[ 0 \leq U(f, P_n) - L(f, P_n) < U_n - L_n = 
        \Bigg [ \int_a^b f + \frac{1}{n} \Bigg ] - 
        \Bigg [ \int_a^b f - \frac{1}{n} \Bigg ] = \frac{2}{n}. \]
    Hence,
    \[ \limit [ U(f, P_n) - L(f, P_n) ] = 2 \limit \frac{1}{n} = 0 \]
    and so \(\{P_n\}\) is an Archimedean sequence. 


    Conversely, suppose we had an Archimedean sequence \(\{P_n\}\) so that 
    \[ \limit [ U(f, P_n) - L(f, P_n) ] = 0 .\]
    Because 
    \[L(f, P_n) \leq \underline{\int_a^b} f \leq \overline{\int_a^b} f \leq U(f, P_n) \]
    by Lemma 6.7, we have (by taking the limit), 
    \[ 0 \leq \overline{\int_a^b} f - \underline{\int_a^b} f \leq \limit [U(f, P_n) - L(f, P_n)] = 0 .\]
    Hence \(\underline{\int_a^b} f = \overline{\int_a^b} f \) and so \(f\)
    is integrable.

    Moreover, Lemma 6.7 shows that \( 0 = \limit U(f, P_n) - \overline{\int_a^b} f \)
    and \(0 = \underline{\int_a^b} f - \limit L(f, P_n) \) and so we get
    \[ \limit L(f, P_n) = \underline{\int_a^b} f = \int_a^b f = \overline{\int_a^b} f = \limit U(f, P_n) .\]
\end{proof}


\begin{example}[6.9]
  Show that
  a monotonically increasing function \(f:[a,b]\to\mb R\) is integrable.
\end{example}

\begin{solution}
    Let \(P_n\) be the regular partition on \([a,b]\). Since \(f\) is 
    monotonically increasing, on a partition interval \([x_{i-1}, x_i]\),
    \(M_i = f(x_{i})\) and \(m_i = f(x_{i-1})\). Then
    \begin{align*}
        \limit [U(f, P_n) - L(f, P_n)] &= \limit \Bigg [ \sumi M_{i}(x_i - x_{i-1}) - \sumi m_i (x_i - x_{i-1}) \Bigg ]\\
            &= \limit \Bigg [ \sumi (M_i - m_i)(x_i - x_{i-1}) \Bigg ] \\
            &= \limit \Bigg [ \sumi (f(x_i) - f(x_{i-1})) \frac{b - a}{n} \Bigg ] \\
            &= \limit \frac{b - a}{n} \Bigg [ \sumi (f(x_i) - f(x_{i-1})) \Bigg ] \\
            &= \limit \frac{b - a}{n} (f(b) - f(a)) \\
            &= 0.
    \end{align*}
    Therefore, by Theorem 6.8, \(f\) is integrable on \([a,b]\).
\end{solution}

\begin{example}[6.11]
  Show that \(\int_0^1 x^2\,dx=\frac{1}{3}\).
\end{example}

\begin{solution}
    Since \(f(x) = x^2\) is monotonically increasing on \([0,1]\), 
    \(f\) is integrable by the above example. Let 
    \(P_{n} = \{x_{0}, x_{1}, ..., x_{n}\}\) be the regular partition 
    on \([0,1]\). Then \(x_i = \frac{i}{n}\) and using the fact that
    \(\sum_{i=1}^{n} i^2 = \frac{n(n +1)(2n+1)}{6}\), we get
    \begin{align*}
        \int_0^1 x^2, dx&= \limit U(f, P_n) \\
                        &= \limit \sumi M_i (x_i - x_{i-1}) \\
                        &= \limit \sumi f(x_{i}) (x_i - x_{i-1}) \\
                        &= \limit \sumi \frac{1}{n} \frac{i^2}{n^2} \\
                        &= \limit \frac{1}{n^3} \sumi i^2 \\
                        &= \limit \frac{1}{n^3} \Bigg [ \frac{n(n+1)(2n+1)}{6} \Bigg ]\\
                        &= \limit \frac{2n^2 + 3n + 1}{6n^2} \\
                        &= \frac{1}{3} .\\
    \end{align*}
\end{solution}

\begin{exercise}[4]
  Prove that for a natural number \(n\),
  \[ \sum_{i=1}^n i = \frac{n(n+1)}{2} .\]
  Then use this fact and the Archimedes-Riemann Theorem to show that
  \(\int_a^b x\,dx=(b^2-a^2)/2\).
\end{exercise}

\begin{solution}
    First, we prove the summation holds by induction on \(n\). If \(n = 1\),
    \(\sum_{i=1}^1 i = 1 = \frac{1(2)}{2}\). Assume this identity holds
    for all natural numbers \(k \leq n\) and now consider \(n + 1\). Then
    \begin{align*}
        \sum_{i=1}^{n+1} i &= \sum_{i = 1}^{n} i + (n + 1) \\
                        &= \frac{n(n+1)}{2} + (n + 1) \\
                        &= \frac{n^2 + 3n + 2}{2} \\
                        &= \frac{ (n+1)((n+1) + 1)}{2} \\
    \end{align*}
    and hence the induction is complete.

    We note that \(f(x) = x\) is monotonically increasing on \(\mb R\) and
    consequently integrable. Thus, for a regular partition \(P_n\) on 
    \([a,b]\), we have
    \begin{align*}
        \int_a^b x\,dx &= \limit U(f, P) \\
                       &= \limit \sumi M_i (x_i - x_{i-1}) \\
                       &= \limit \sumi x_i \frac{b - a}{n} \\
                       &= \limit \sumi \Bigg (a + i \frac{b-a}{n} \Bigg ) \frac{b-a}{n} \\
                       &= \limit \frac{b-a}{n} \Bigg [ \sumi a + \frac{b-a}{n} \sumi i \Bigg ] \\
                       &= \limit \frac{b-a}{n} \Bigg [ na + \frac{b-a}{n} \cdot \frac{n(n+1)}{2} \Bigg ] \\
                       &= \limit [(ab - a^2) + \frac{(b-a)^2 (n+1)}{2n}] \\
                       &= ab - a^2 + \frac{(b-a)^2}{2} \\
                       &= \frac{2ab - 2a^2 + b^2 - 2ab + a^2}{2} \\
                       &= \frac{b^2 - a^2}{2} .\\
    \end{align*}
\end{solution}


\begin{exercise}[6b]
  Use the Archimedes-Riemann Theorem to show that for \(0\leq a<b\),
  \[
    \int_a^b x^2\,dx = \frac{b^3-a^3}{3}
  .\]
\end{exercise}

\begin{solution}
    Generalizing from Example 6.11, 
     \begin{align*}
        \int_a^b x^2, dx&= \limit U(f, P_n) \\
                        &= \limit \sumi M_i (x_i - x_{i-1}) \\
                        &= \limit \sumi f(x_{i}) (x_i - x_{i-1}) \\
                        &= \limit \sumi \frac{b - a}{n} (a + \frac{b-a}{n} i)^2 \\
                        &= \limit \frac{b - a}{n} \Bigg [ a^2 \sumi 1 + 2a \frac{b-a}{n} \sumi i + \frac{(b-a)^2}{n^2} \sumi i^2 \Bigg ] \\ 
                        &= \limit \frac{b - a}{n} \Bigg [ na^2 + a(b - a)(n + 1) + \frac{ (n + 1)(2n + 1)(b - a)^2}{6n} \Bigg ] \\
                        &= \limit a^2 (b - a) + \limit a(b - a)^2 \frac{n+1}{n} + \limit \frac{ (2n^2 + 3n + 1) (b - a)^3}{6n^2} \\
                        &= a^2 (b - a) + a(b - a)^2 + \frac{(b - a)^3}{3} \\
                        &= \frac{1}{3} \Bigg [ (3a^2 b - 3a^3) + (3ab^2 - 6a^2 b + 3a^3) + (b^3 - 3ab^2 + 3a^2 b - a^3) \Bigg ] \\  
                        &= \frac{b^3 - a^3}{3} \\
    \end{align*}
\end{solution}


\begin{exercise}[9]
  Suppose that the functions \(f:[a,b]\to\mb R\) and
  \(g:[a,b]\to\mb R\) are integrable. Show that there is a sequence
  \(\{P_n\}\) of partitions of \([a,b]\) that is an Archimediean sequence
  of partitions for \(f\) on \([a,b]\) and also an Archimedean sequence
  of partitions for \(g\) on \([a,b]\).
\end{exercise}
\begin{solution}
    By the Archimedes-Riemann Theorem, there exists Archimedean sequences
    \(Q_n\) and \(R_n\) for \(f\) and \(g\), respectively, such that 
    \( \limit [U(f, Q_n) - L(f, Q_n)] = 0\) and \(\limit [U(g, R_n) - L(g, R_n)] = 0\).
    For each \(n\), define \(P_n = Q_n \cup R_n\). The Refinement lemma implies
    \[ 0 = \limit [U(f, Q_n) - L(f, Q_n)] \geq \limit [U(f, P_n) - L(f, P_n)] \geq 0 \]
    and 
    \[ 0 = \limit [U(g, R_n) - L(g, R_n)] \geq \limit [U(g, P_n) - L(g, P_n)] \geq 0 .\]
    Therefore, \(\{P_n\}\) is an Archimedean sequence for \(f\) and \(g\).
\end{solution}




\section{Additivity, Monotonicity, and Linearity}


\begin{theorem}[6.12, Additivity over Intervals]
  Let \(f:[a,b]\to\mb R\) be integrable on \([a,b]\) and let \(c\in(a,b)\).
  Then \(f\) is integrable on \([a,c]\) and \([c,b]\), and furthermore
  \[
    \int_a^b f = \int_a^c f + \int_c^b f
  .\]
\end{theorem}

\begin{proof}
    By the Archimedes-Riemann Theorem, there exists an Archimedean sequence
    \(Q_n\) of \(f\) on \([a,b]\). Define \(P_n = Q_n \cup \{c\}\) for
    every \(n \in \mb N\). This refinement is also an Archimedean sequence
    of \(f\) on \([a,b]\). Next, define \(R_n = P_n \cap [a, c]\) and
    \(S_n = P_n \cap [b, c]\) for every \(n \in \mb N\). Since 
    \(U(f, P_n) = U(f, R_n) + U(f, S_n)\) and \(L(f, P_n) = L(f, R_n) + L(f, S_n)\),
    we have
    \begin{align*}
        0 &= \limit [ U(f, P_n) - L(f, P_n) ] \\
          &= \limit [ (U(f, R_n) + U(f, S_n)) - (L(f, R_n) + L(f, S_n)) ] \\
          &= \limit [ U(f, R_n) - L(f, R_n) ] + \limit [ U(f, S_n) - L(f, S_n) ] .\\
    \end{align*}
    Since the terms within the limits are nonnegative, both limits go 
    to zero. Hence, \(R_n\) and \(S_n\) are Archimedean sequences of \(f\)
    on \([a,c]\) and \([c, b]\), respectively. Thus, \(f\) is integrable
    on \([a,c]\) and \([c, b]\). Moreover,
    \begin{align*}
        \int_a^b f &= \limit U(f, P_n) \\
                   &= \limit (U(f, R_n) + U(f, S_n)) \\
                   &= \limit U(f, R_n) + \limit U(f, S_n) \\
                   &= \int_a^c f + \int_c^b f . \\
    \end{align*}
\end{proof}


\begin{theorem}[6.13, Monotonicity of the Integral]
  Suppose \(f,g:[a,b]\to\mb R\) are integrable and that \(f(x)\leq g(x)\)
  for all \(x\in[a,b]\). Then
  \[
    \int_a^b f \leq \int_a^b g
  .\]
\end{theorem}

\begin{proof}
    By the Archimedes-Riemann theorem, we have Archimedean sequences 
    \(\{Q_n\}\) and \(\{R_n\}\)for \(f\) and \(g\) on \([a,b]\), respectively. 
    For each \(n \in \mb N\), define \(P_n = Q_n \cup R_n\). Since \(\{P_n\}\)
    refines both sequences, it is an Archimedean sequence of both \(f\) and \(g\)
    on \([a,b]\). Consequently,
    \begin{align*}
        \int_a^b f &= \limit U(f, P_n) \\
                   &\leq \limit U(g, P_n) \\
                   &= \int_a^b g \\
    \end{align*}
    following from the fact that \(f(x) \leq g(x)\) implies \(U(f, P) \leq U(g, P)\)
    (by definition) for any partition \(P\).
\end{proof}


\begin{lemma}[6.14]
  Let \(f,g:[a,b]\to\mb R\) be bounded and let \(P\) partition \([a,b]\).
  Then
  \[
    L(f,P)+L(g,P)\leq L(f+g,P)
      \text{~~and~~}
    U(f+g,P)\leq U(f,P)+U(g,P)
  .\]
  Moreover, for any number \(\alpha\),
  \[
    U(\alpha f,P)=\alpha U(f,P)
      \text{~~and~~}
    L(\alpha f,P)=\alpha L(f,P)
      \text{~~if~}
    \alpha\geq 0
  \]
  \[
    U(\alpha f,P)=\alpha L(f,P)
      \text{~~and~~}
    L(\alpha f,P)=\alpha U(f,P)
      \text{~~if~}
    \alpha< 0
  .\]
\end{lemma}

\begin{proof}
    Let \(B([a,b])\) be the set of bounded real functions on \([a,b]\).
    For partition \(P = \{x_0, x_1, ..., x_n\}\), denote \(I_i = [x_{i-1}, x_i]\) 
    for \(i \in [n]\). Then, define functions: 
    \[M_i, m_i : B([a,b]) \rightarrow \mb R \word{by}\]
    \[M_i = \sup \{ h(x) | x \in I_i \} \word{and} m_i = \inf \{ h(x) | x \in I_i \} . \]
    Then, for all \(x \in I_i\), we have
    \[ m_i(f) + m_i(g) \leq f(x) + g(x) \leq M_i(f) + M_i(g) .\]
    This shows that \(m_i(f) + m_i(g)\) is a lower bound of \(f + g\) on \(I_i\),
    and similarly for \(M_i(f) + M_i(g)\) (upper bound). Hence, by definition, and
    noting that the pointwise sum of bounded functions is bounded, we get
    \[ m_i(f) + m_i(g) \leq m_i(f + g) \word{and} M_i(f + g) \leq M_i(f) + M_i(g) .\]
    Therefore, this implies that 
    \[ L(f, P) + L(g, P) \leq L(f+g, P) \word{and} U(f + g, P) \leq U(f, P) + U(g, P) .\]

   Next, by Exercise 4, we know that 
   \[ M_i(\alpha f) = \alpha M_i(f) \word{and} m_i(\alpha f) = \alpha m_i(f) \word{if} \alpha \geq 0 \]
   and so \(U(\alpha f, P) = \alpha U(f, P)\) and \(L(\alpha f, P) = \alpha L(f, P) \).
   Likewise,
   \[ M_i(\alpha f) = \alpha m_i(f) \word{and} m_i(\alpha f) = \alpha M_i(f) \word{if} \alpha \leq 0 \]
   and thus \(U(\alpha f, P) = \alpha L(f, P)\) and \(L(\alpha f, P) = \alpha U(f, P) \).
\end{proof}


\begin{theorem}[6.15, Linearity of the Integral]
  Let \(f,g:[a,b]\to\mb R\) be integrable. Then for any two numbers
  \(\alpha,\beta\), the function \(\alpha f+\beta g:[a,b]\to\mb R\) is
  integrable and
  \[
    \int_a^b[\alpha f+\beta g]=\alpha\int_a^b f + \beta\int_a^b g
  .\]
\end{theorem}

\begin{proof}
    It suffices to show two things: first, for any integrable function 
    \(h: [a,b] \to \mb R\) and any number \(\alpha\), the function 
    \(\alpha h\) is integrable on \([a,b]\) and 
    \[ \int_a^b \alpha h = \alpha \int_a^b h .\]
    Second, for integrable functions \(f, g\) on \([a,b]\), the function
    \(f + g\) is integrable and 
    \[ \int_a^b [f + g] = \int_a^b f + \int_a^b g .\]
    From these, we get
    \begin{align*} 
        \int_a^b[ \alpha f + \beta g ] &= \int_a^b \alpha f + \int_a^b \beta g \\
                                       &= \alpha \int_a^b f + \beta \int_a^b g .\\
    \end{align*}


    We now prove the first fact. Let \(\{P_n\}\) be an Archimedean sequence
    of \(h\). Applying the previous lemma, regardless of the
    value of \(\alpha\) we have
    \[ U(\alpha h, P_n) - L(\alpha h, P_n) = |\alpha| [U(h, P_n) - L(h P_n)] \]
    and so taking the limit tells us that \(\{P_n\}\) is an Archimedean 
    sequence of \(\alpha h\). Now, for \(\alpha \geq 0\),
    \[ \int_a^b \alpha h = \limit U(\alpha h, P_n) = \alpha \limit U(h, P_n) = \alpha \int_a^b h\]
    while for \(\alpha \leq 0\),
    \[ \int_a^b \alpha h = \limit U(\alpha h, P_n) = \alpha \limit L(h, P_n) = \alpha \int_a^b h .\]
    This gives us our result.


    Now we prove the second fact. From the proof of Theorem 6.13, we know
    there exists an Archimedean sequence \(\{P_n\}\) common to both \(f\)
    and \(g\). Then, applying the previous lemma, we get
    \begin{align*}
        0 &\leq  \limit [ U(f + g, P_n) - L(f + g, P_n) ] \\ 
          &\leq \limit [ [U(f, P_n) + U(g, P_n)] - [L(f, P_n) + L(g, P_n)]] \\
          &= \limit [U(f, P_n) - L(f, P_n)] + \limit [U(g, P_n) - L(g, P_n)] \\
          &= 0 .
    \end{align*}
    This shows that \(\{P_n\}\) is an Archimedean sequence of \(f+g\). It follows
    that
    \begin{align*} 
        \int_a^b f + \int_a^b g &= \limit U(f, P_n) + \limit U(g, P_n) \\
                                &= \limit [U(f, P_n) + U(g, P_n)] \\
                                &\geq \limit [U(f + g, P_n)] \\
                                &= \int_a^b [f + g] \\
                                &= \limit [ L(f + g, P_n) ] \\
                                &\geq \limit [L(f, P_n) + L(g, P_n) ] \\
                                &= \limit L(f, P_n) + \limit L(g, P_n) \\
                                &= \int_a^b f + \int_a^b g \\
    \end{align*}
    and so \(\int_a^b [f + g] = \int_a^b f + \int_a^b g \).

\end{proof}


\begin{exercise}[1]
  Suppose that the functions \(f,g,f^2,g^2,fg\) are integrable on \([a,b]\).
  Prove that \((f-g)^2\) is also integrable on \([a,b]\) and that
  \(\int_a^b(f-g)^2\geq0\). Use this to prove that
  \[
    \int_a^b fg
      \leq
    \frac{1}{2}\left[
      \int_a^b f^2 + \int_a^b g^2
    \right]
  .\]
\end{exercise}

\begin{solution}
    Since \((f - g)^2 = f^2 - 2fg + g^2\) and each term on the RHS
    is integrable, by linearity, \((f-g)^2\) is integrable. Also,
    since \((f - g)^2 \geq 0\) (the zero function) and \(\int_a^b 0 = 0\),
    we have from the monotonicity property that
    \[ \int_a^b f^2 - 2 \int_a^b fg + \int_a^b g^2 = \int_a^b (f - g)^2 \geq \int_a^b 0 = 0 .\]
    Hence, \( \int_a^b f^2 + \int_a^b g^2 \geq 2 \int_a^b fg \) and thus
    \[ \frac{1}{2} \left[ \int_a^b f^2 + \int_a^b g^2 \right] \geq \int_a^b fg .\] 
\end{solution}

\begin{exercise}[4]
  Suppose that \(S\) is a nonempty bounded set of numbers and that \(\alpha\)
  is a number. Define \(\alpha S\) to be the set \(\{\alpha x:x\in S\}\).
  Prove that
  \[
    \sup\alpha S=\alpha\sup S
      \text{~~and~~}
    \inf\alpha S=\alpha\inf S
      \text{~~if~}
    \alpha\geq 0
  \]
  while
  \[
    \sup\alpha S=\alpha\inf S
      \text{~~and~~}
    \inf\alpha S=\alpha\sup S
      \text{~~if~}
    \alpha< 0
  .\]
\end{exercise}

\begin{solution}
    Since \(S\) is a nonempty, bounded set of real numbers, \(b = \sup S\)
    exists by the Completeness Axiom. Next, because \(\alpha\) is a fixed nonnegative
    number, \(\alpha S\) is also nonempty and bounded and so a supremum for this
    set exists. Claim: \( \sup \alpha S = \alpha b \). In the case where \(\alpha = 0\),
    then \(\alpha S = \{0\}\) and so \(\sup \alpha S = 0 = \alpha \sup S\),
    trivially. Otherwise, if \(\alpha > 0\), note that \(b \geq s \) for all \(s \in S\)
    and so \(\alpha b \geq \alpha s\). Since all elements of \(\alpha S\) take the 
    form of \(\alpha s\) for some \(s \in S\), we have that \(\alpha b\) is an upper 
    bound of \(\alpha S\). Now suppose that we have \(\alpha b > x\). 
    Dividing by \(\alpha\), we get \(b > \alpha^{-1} x \). Then, by definition,
    \(\alpha^{-1} x\) is not an upper bound of \(S\) and so there exists \(u \in S\)
    such that \(b \geq u > \alpha^{-1} x\). This implies
    \[ \alpha b \geq \alpha u > x \]
    and so \(x\) cannot be an upper bound of \(\alpha S\). Thus, \(\alpha b = \sup \alpha S\).

    Last, consider when \(\alpha < 0\) and let \(c = \inf S\). For all \(s \in S\),
    \(s \geq c\) and so multiplying by \(\alpha\) gives us \(\alpha s \leq \alpha c\). 
    By our earlier reasoning \(\alpha c\) is an upper bound of \(\alpha S\), so
    suppose \(\alpha c > x\). Divide by \(\alpha\) to get \(c < \alpha^{-1}x\).
    By definition of the infimum, there exists \(v \in S\) so that
    \(v < \alpha^{-1}x\). Multiplying by \(\alpha\) yields \(\alpha v > x \)
    and so \(x\) cannot be an upper bound of \(\alpha S\). Thus, \(\sup \alpha S = \alpha c\).

    Repeat all of the above with the set \(-S\) and note that \(\sup -S = \inf S\)
    to derive the equalities for the infimums of \(\alpha S\).
\end{solution}


\begin{exercise}[6]
  Suppose that \(f:[a,b]\to\mb R\) is bounded and let \(a<c<b\). Prove that if
  \(f\) is integrable on both \([a,c],[c,b]\), then it is integrable on
  \([a,b]\).
\end{exercise}

\begin{solution}
    By assumption, there exists Archimedean sequences \(\{Q_n\}\) and \(\{R_n\}\) on
    \([a, c]\) and \([c, d]\), respectively, such that \(\limit U(f, Q_n) = \int_a^c f\)
    and \(\limit U(f, R_n) = \int_c^b f\). For each \(n \in \mb N\), define the
    set \(P_n : Q_n \cup R_n\) (with say \(Q_n = \{q_0, ..., q_k\}\) and 
    \(R_n = \{r_0, ..., r_m\}\)). Then \(p_0 = q_0 = a\), \(p_k = q_k = c = r_0\),
    and \(p_{k+m} = r_m = b\) and since \(Q_n \cap R_n = \{c\}\), \(P_n\) is 
    a partition of \([a,b]\). Next, note that \(U(f, P_n) = U(f, Q_n) + U(f, R_n)\)
    and \(L(f, P_n) = L(f, Q_n) + L(f, R_n)\). It then follows that
    \begin{align*}
        \limit [ U(f, P_n) - L(f, P_n) ] &= \limit [ (U(f, Q_n) + U(f, R_n) - (L(f, Q_n) + L(f, R_n) ] \\
                &= \limit [U(f, Q_n) - L(f, Q_n)] + \limit [U(f, R_n) - L(f, R_n)] \\
                &= 0 + 0 \\
                &= 0 \\
    \end{align*}
    and so \(\{P_n\}\) is an Archimedean sequence. Therefore, \(f\) is integrable on \([a,b]\).
\end{solution}




\section{Continuity and Integrability}


\begin{lemma}[6.17]
  Let the function \(f:[a,b]\to\mb R\) be continuous let \(P\) partition
  its domain. Then there is a partition interval of \(P\) that contains two
  points \(u,v\) for which the following estimate holds:
  \[
    0
      \leq
    U(f,P)-L(f,P)
      \leq
    [f(v)-f(u)][b-a]
  .\]
\end{lemma}

\begin{proof}
    Let \(P = \{x_0, x_1, ..., x_n\}\) be a partition of \([a,b]\). For
    \(i \in [n]\), consider the restriction of \(f\) to \([x_{i-1}, x_i]\).
    Since \(f\) is continuous, so is its restriction. Morever, the partition
    interval is a closed, bounded interval. Thus, by the Extreme value theorem,
    there exists points \(u_i, v_i \in [x_{i-1}, x_i]\) such that
    \(f(u_i) = \min f([x_{i-1}, x_i])\) and \(f(v_i) = \max f([x_{i-1}, x_i])\).
    Now let \[k = \word{argmax}_{i \in [n]} \{ f(v_i) - f(u_i) \} \]
    and set \(u = u_k\) and \(v = v_k\). Hence,
    \begin{align*}
        0  &\leq U(f, P) - L(f, P) \\
           &= \sumi (M_i - m_i)(x_i - x_{i-1}) \\
           &= \sumi (f(v_i) - f(u_i))(x_i - x_{i-1}) \\
           &\leq \sumi (f(v) - f(u))(x_i - x_{i-1}) \\
           &= [f(v) - f(u)] \sumi (x_i - x_{i-1}) \\
           &= [f(v) - f(u)] [b - a] \\
    \end{align*}
\end{proof}


\begin{theorem}[6.18]
  A continuous function on a closed bounded interval is integrable.
\end{theorem}

\begin{proof}
    Let \(f\) be a continuous function on \([a,b]\), and
    let \(\{P_n\}\) be a sequence of partitions where \(\limit \text{gap~} P_n = 0\)
    (for instance, regular partitions). For each \(n \in \mb N\),
    apply the preceding lemma to obtain points \(u_n\) and \(v_n\). Since the two points
    belong to the same partition interval, we have \(|v_n - u_n| \leq \text{gap~} P_n\).
    It follows that
    \[ 0 = \limit - \text{gap~} P_n \leq \limit [v_n - u_n] \leq \limit \text{gap~} P_n = 0. \]
    Next, we note that a continuous function on a closed, bounded interval
    is uniformly continuous and consequently, \(\limit [f(v_n) - f(u_n)] = 0\).
    By the previous lemma,
    \[ 0 \leq \limit [ U(f, P_n) - L(f, P_n)] \leq \limit [f(v_n) - f(u_n)][b - a] = 0 \]
    and so \(\{P_n\}\) is an Archimedean sequence of \(f\). Thus, \(f\)
    is integrable.
\end{proof}


\begin{theorem}[6.19]
  Suppose \(f:[a,b]\to\mb R\) is bounded on \([a,b]\) and continuous on
  \((a,b)\). Then \(f\) is integrable on \([a,b]\) and the value of
  \(\int_a^b f\) does not depend on the values of \(f\) at the endpoints
  of \([a,b]\).
\end{theorem}

\begin{proof}
    Construct sequences \(\{a_n\}\) and \(\{b_n\}\) such that 
    \[ a < a_n < b_n < b \word{for all} n ,\]
    \(\limit a_n = a\), and \(\limit b_n = b\). For each \(n \in \mb N\),
    the restriction of \(f\) to \([a_n, b_n]\) is continuous and hence
    integrable. This means there exists a corresponding Archimedean
    sequence and so we can choose a partition \(P_n^*\) of \([a_n, b_n]\)
    such that 
    \[ 0 \leq U(f, P_n^*) - L(f, P_n^*) \leq \frac{1}{n} \]
    (by definition of the limit of a sequence). We then define
    \(P_n = P_n^* \cup \{a, b\}\) so that \(P_n\) is a partition of 
    \([a,b]\). Let \(A_n = U(f, \{a, a_n\}) - L(f, \{a, a_n\})\)
    and \(B_n = U(f, \{b_n, b\}) - L(f, \{b_n, b\})\). Then
    \[ U(f, P_n) - L(f, P_n) = U(f, P_n^*) - L(f, P_n^*) + A_n + B_n .\]

    Because \(f\) is bounded, there exists \(M\) such that
    \(-M \leq f(x) \leq M\) for \(x \in [a,b]\). Consequently, by Lemma 6.1,
    \[ 0 \leq A_n \leq [M - (-M)][a_n - a] = 2M[a_n - a] \word{and} 0 \leq B_n \leq 2M[b_n - b] .\]
    
    This means that 
    \[ 0 \leq U(f, P_n) - L(f, P_n) \leq [U(f, P_n^*) - L(f, P_n^*)] + 2M[a_n - a] + 2M[b_n - b] .\]

    Since we know that 
    \[0 \leq \limit [U(f, P_n^*) - L(f, P_n^*)] \leq \limit \frac{1}{n} = 0 ,\]
    \(\limit [a_n - a] = 0\), and \(\limit [b_n - b] = 0\), we get
    \[ \limit [U(f, P_n) - L(f, P_n)] = 0 .\]

    Therefore, \(\{P_n\}\) is an Archimedean sequence for \(f\) on \([a,b]\),
    and \(f\) is integrable.

    
    Moreover, 
    \[ \limit [ U(f, P_n) - U(f, P_n^*) ] = \limit [ U(f, \{a, a_n\}) +  U(f, \{b_n, b\}) ] \leq \limit M[a_n - a] + \limit M[b_n - b] = 0 .\]
    Thus,
    \[ \int_a^b f = \limit U(f, P_n) = \limit U(f, P_n^*) \]
    and so the integral is indepedent of the values of \(f(a)\) and \(f(b)\).

\end{proof}

\begin{exercise}[1]
  Determine whether each of the following statements is true or false, and
  justify your answer.
  \begin{enumerate}[(a)]
    \item If \(f:[a,b]\to\mb R\) is integrable and \(\int_a^b f=0\), then
      \(f(x)=0\) for all \(x\in[a,b]\).
    \item If \(f:[a,b]\to\mb R\) is integrable, then \(f\) is continuous.
    \item If \(f:[a,b]\to\mb R\) is integrable and \(f(x)\geq0\) for all
      \(x\in[a,b]\), then \(\int_a^b f\geq 0\).
    \item A continuous function \(f:(a,b)\to\mb R\) defined on an open interval
      \((a,b)\) is bounded.
    \item A continuous function \(f:[a,b]\to\mb R\) defined on a closed interval
      \([a,b]\) is bounded.
  \end{enumerate}
\end{exercise}
\begin{solution}
  \begin{enumerate}[(a)]
    \item False. Define \(f\) to be identically zero on \((a,b)\) and
          \(f(a) = f(b) = 1\). Then \(f\) on \([a,b]\) is not identically zero,
          but \(\int_a^b f = 0\) by Theorem 6.19.
    \item False. Define \(f: [a, b] \rightarrow \mb R\) by
         \[ f(x) = \begin{cases} 
                 0 &\word{if} x \leq (b - a)/2 \\
                 1 &\word{if} x > (b - a)/2 \\
             \end{cases} \]
         Since \(f\) is monotonically increasing, it is integrable. However,
         \(f\) is not continuous at \(x = 1/2\).
    \item True. Since \(f(x) \geq 0(x) = 0\) on \([a,b]\), by monotonicity,
        \[ \int_a^b f \geq \int_a^b 0 = 0 .\]
    \item False. The function \(f(x) = 1/x\) on \((0,1)\) is continuous but
        not bounded above. Indeed, suppose there exists a number \(M > 0\)
        such that \(M \geq f(x)\) within this interval. Choose any point
        greater than \(1 / M\) for the contradiction.
    \item True. This is a consequence of the Extreme value theorem.
  \end{enumerate}
\end{solution}


\begin{exercise}[5]
  Suppose that the continuous function \(f:[a,b]\to\mb R\) has the property
  \[
    \int_c^d f\leq 0
      \text{~~whenever~}
    a\leq c<d\leq b
  .\]
  Prove that \(f(x)\leq 0\) for all \(x\in[a,b]\). Is this true if we only
  require integrability of the function?
\end{exercise}

\begin{solution}
    Suppose on the contrary that there exists \(x \in [a,b]\) such that
    \(f(x) > 0\). Since \(f\) is continuous, \(f\) obtains a maximum \(f(x_0)\)
    by the Extreme value theorem. By our assumption, it follows that 
    \(f(x_0) > 0\). Now let \(\epsilon = f(x_0)\) so that, by continuity,
    there exists \(\delta > 0\) such that 
    \[ |x - x_0| < \delta \word{implies} |f(x_0) - 0| < \epsilon .\]
    This implies that we can find an open neighborhood \(I\) contained
    in \([a,b]\) such that \(x \in I\) implies \(f(x) > 0\). (Even
    in the case where \(x_0\) is an endpoint, we can simply omit
    \(x_0\) from consideration  and use the remaining open interval).
    Consequently, \(f\) is integrable on \(I\) (by Theorem 6.19) 
    and its integral is positive (by monotonicity). However, since 
    the bounds of \(I\) are within \([a,b]\), we contradict the 
    property that \(\int_c^d f \leq 0\) whenever \(a \leq c < d \leq b\).
    Hence, \(f(x) \leq 0\) for all \(x \in [a,b]\).

    Clearly, the statement becomes false if we only require the
    integrability of \(f\). For instance, define \(f:[0,1] \rightarrow \mb R\)
    by \[ f(x) = \begin{cases} 1  &\word{if} x < 1/3 \\
                              -1 &\word{if} x \geq 1/3 \\
                \end{cases}. \]
    Then, \(\int_0^1 f = -\frac{1}{3} \leq 0 \) but not all
    \(f(x) \leq 0\).
\end{solution}


\begin{exercise}[6]
  Suppose that \(f:[0,1]\to\mb R\) is continuous and that \(f(x)\geq 0\) for
  all \(x\in[0,1]\). Prove that \(\int_0^1 f>0\) if and only if there is a
  point \(x_0\in[0,1]\) at which \(f(x_0)>0\).
\end{exercise}

\begin{solution}
    We'll prove the contrapositives of the equivalency:
    \[ \int_0^1 f \leq 0 \Leftrightarrow \word{for all} x \in [a,b], f(x) \leq 0 \]
    
    By assumption, \(f(x) \geq 0\) for all \(x \in [0, 1]\)
    implies \(\int_0^1 f \geq 0\). Hence, if we suppose that
    \(\int_0^1 f \leq 0\), then \(\int_0^1 f = 0\), which can
    only happen if \(f(x) = 0\) for all \(x \in [a,b]\). This
    clearly means \(f(x) \leq 0 \) for all \(x \in [a,b]\).
    The converse follows from monotonicity.
\end{solution}




\section{The First Fundamental Theorem: Integrating Derivatives}


\begin{lemma}[6.21]
  Suppose \(f:[a,b]\to\mb R\) is integrable and that the number \(A\) has
  the property that for every \(P\) partitioning \([a,b]\),
  \[
    L(f,P) \leq A \leq U(f,P)
  .\]
  Then
  \[
    \int_a^b f = A
  .\]
\end{lemma}

\begin{proof}
    By assumption, \(A\) is a lower bound for \(\{U(f, P)\}_P\)
    and an upper bound for \(\{L(f, P)\}_P\) and so by definition
    \[ \underline{\int_a^b f} \leq A \leq \overline{\int_a^b f} .\]
    But since \(f\) is integrable, 
    \[ \underline{\int_a^b f} = \int_a^b f = \overline{\int_a^b f} \]
    and so 
    \[\underline{\int_a^b f } = \int_a^b f \leq A \leq \int_a^b f = \overline{\int_a^b f} \] 
    implies \(\int_a^b f = A\).
\end{proof}


\begin{theorem}[6.22, The First Fundamental Theorem: Integrating Derivatives]
  Let \(F:[a,b]\to\mb R\) be continuous on \([a,b]\) and differentiable on
  \((a,b)\). Moreover, suppose that its derivative
  \(F':(a,b)\to\mb R\) is both continuous and bounded. Then
  \[
    \int_a^b F'(x)~dx
      =
    F(b)-F(a)
  .\]
\end{theorem}

\begin{proof}
    We must show two things: (1) that the integral exists, and (2) that it
    is equal to \(F(b) - F(a)\). First, any extension of \(F'\) to \([a,b]\)
    remains continuous on \((a,b)\) and bounded on \([a,b]\). Thus, by Theorem
    6.19, any extension of \(F'\) is integrable. Further, every such integral
    has the same value and so \(\int_a^b F'\) is unambiguously defined for
    our original \(F'\).
    
    Second, we show that for every partition \(P = \{x_0, x_1, ..., x_n\}\) 
    of \([a,b]\),
    \[ L(F', P) \leq F(b) - F(a) \leq U(F', P) .\]
    For \(i \in [n]\), the restriction of \(F\) to \([x_{i-1}, x_i]\) is
    continuous on this closed interval, and differentiable on \((x_{i-1}, x_i)\).
    Hence, by the Mean Value Theorem, there exists a point \(c_i \in (x_{i-1}, x_i)\)
    so that 
    \[ F(x_i) - F(x_{i-1}) = F'(c_i)(x_i - x_{i-1}) .\]
    Consequently, \(m_i \leq F'(c_i) \leq M_i\) (where \(m_i\) and \(M_i\) are the infimum
    and supremum of the restriction of \(F'\), respectively). Then, multiplying this 
    inequality by \(x_i - x_{i-1}\) yields
    \[ m_i (x_i - x_{i-1}) \leq F'(c_i)(x_i - x_{i-1}) = F(x_i) - F(x_{i-1}) \leq M_i (x_i - x_{i-1}) .\]
    Taking the sum across all partition intervals, we have
    \[ L(F', P) = \sumi m_i(x_i - x_{i-1}) \leq \sumi [F(x_i) - F(x_{i-1}) ] = F(b) - F(a) \leq \sumi M_i (x_i - x_{i-1}) = U(F', P) \] 
    Thus, by the previous lemma
    \[ \int_a^b F'(x) dx = F(b) - F(a) .\]
\end{proof}


\begin{exercise}[1]
  Let \(m,b\) be positive numbers. Find the value of \(\int_0^1 mx+b ~dx\)
  in the following three ways:
  \begin{enumerate}[(a)]
    \item Using elementary geometry, interpreting the integral as an area.
    \item Using upper and lower Darboux sums based on regular partitions of
      the interval \([0,1]\) and using the Archimedes-Riemann Theorem.
    \item Using the First Fundamental Theorem (Integrating Derivatives).
  \end{enumerate}
\end{exercise}

\begin{solution}
    \begin{enumerate}[(a)]
        \item Geometrically, the area between \(mx + b\) and the x-axis consists
              of two shapes: a rectangular base and a triangle on top. The area
              of the rectangle is \[A_r = l \times w = 1 \times b = b\] and the area
              of the triangle is 
              \[ A_t = \frac{1}{2} b \times h = \frac{1}{2}(1 \times (m + b - b)) = \frac{1}{2} m .\]
              Consequently,
              \[ \int_0^1 mx + b ~dx = A_r + A_t = b + \frac{1}{2} m .\]
        \item The function \(f(x) = mx + b\) is monotonically increasing on \([0,1]\) 
              and therefore integrable by the Archimedes-Riemann Theorem. Using a sequence
              of regular partitions, we get
              \begin{align*}
                  \int_0^1 mx + b ~dx &= \limit U(f, P_n)\\
                                      &= \limit \sumi M_i \frac{1}{n} \\
                                      &= \limit \sumi (mx_i + b) \frac{1}{n} \\
                                      &= \limit [ m \sumi \frac{i}{n^2} + \frac{b}{n} \sumi 1 ] \\
                                      &= \limit [ m \frac{(n+1)}{2n} + b ] \\
                                      &= \frac{1}{2}m + b \\
              \end{align*}

          \item By linearity, \[\int_0^1 mx + b ~dx  = m \int_0^1 x ~dx + b \int_0^1 1 ~dx .\]
                Clearly, \(x\) and \(1\) are continuous and bounded on \([0, 1]\). Furthermore,
                we know that \(\frac{d}{dx} \frac{x^2}{2} = x\) and \(\frac{d}{dx} x = 1\). Thus,
                by the First Fundamental Theorem of Calculus
                \[ \int_0^1 x ~dx = \frac{1^2}{2} - \frac{0^2}{2} = \frac{1}{2} \]
                and
                \[ \int_0^1 1 ~dx = 1 - 0 = 1 .\]
                Putting it altogether, we get
                \[ \int_0^1 mx + b ~dx = m \int_0^1 x ~dx + b \int_0^1 1 ~dx = m \frac{1}{2} + b .\]
    \end{enumerate}
\end{solution}


\begin{exercise}[5]
  The monotonicity property of the integral implies that if the functions
  \(g,h:[0,\infty)\to\mb R\) are continuous and \(g(x)\leq h(x)\) for all
  \(x\geq 0\), then
  \[
    \int_0^x g\leq \int_0^x h
    \text{~~ for all~} x\geq 0
  .\]
  Use this and the First Fundamental Theorem to show that each of the following
  inequalities implies the next:
  \[
    \cos x \leq 1
    \text{~~ if~} x\geq 0
  .\]
  \[
    \sin x \leq x
    \text{~~ if~} x\geq 0
  .\]
  \[
    1-\cos x \leq \frac{x^2}{2}
    \text{~~ if~} x\geq 0
  .\]
  \[
    x-\sin x \leq \frac{x^3}{6}
    \text{~~ if~} x\geq 0
  .\]
  \[
    x-\frac{x^3}{6} \leq \sin x \leq x
    \text{~~ if~} x\geq 0
  .\]

  (For this problem, you may assume that the sine and cosine functions
  are differentiable functions with the properties
  \(\sin(0)=0\), \(\cos(0)=1\), \(\frac{d}{dx}[\sin(x)]=\cos(x)\),
  and \(\frac{d}{dx}[\cos(x)]=-\sin(x)\).)
\end{exercise}
\begin{solution}

\end{solution}

\begin{solution}
    By assumption (of the first inequality), \(\text{cos~}x\) is bounded
    on \([0, \infty)\). We also know that it is continuous. Moreover,
    we know that \(\frac{d}{dx} \text{sin~} x = \text{cos~} x\) on \(\mb R\), and
    that \(\text{sin}(0) = 0\). These facts, combined with knowing that
    \(\frac{d}{dx} x^n = nx^{n-1}\) for \(n \geq 1\), we get
    \[ \text{sin}(x) = \text{sin}(x) - \text{sin}(0) = 
            \int_0^x \text{cos~} x ~dx \leq  \int_0^x 1 ~dx = x \]
    for \(x \geq 0\). 

    Next, since \(\text{sin}(x + 2\pi) = \text{sin}(x)\) for \(x \in \mb R\) and
    \(\text{sin}(x)\) is continuous on \([0, 2\pi]\), \(\text{sin}(x)\) is 
    bounded on \([0, 2\pi]\) (by the Extreme Value Theorem), and consequently,
    \(\text{sin}(x)\) is bounded on \([0, \infty)\). We know that
    \(\frac{d}{dx} -\text{cos}(x) = \text{sin}(x) \). Hence, we can apply the
    first Fundamental Theorem of Calculus to obtain
    \[ 1 - \text{cos}(x) = [ -\text{cos}(x) - (-\text{cos}(0)) ] 
        = \int_0^x \text{sin~} x ~dx \leq \int_0^x x ~dx 
    = \frac{x^2}{2} - \frac{0^2}{2} = \frac{x^2}{2} \]
    for \(x \geq 0\).

    And again
    \[ x - \text{sin}(x) = \int_0^x [x - \text{sin}(x) ] ~ dx \leq \int_0^x \frac{x^2}{2} ~ dx = \frac{1}{2}[\frac{x^3}{3}] = \frac{x^3}{6}. \]

    Rearranging this inequality with the second inequality gives us
    \[ x - \frac{x^3}{6} \leq \text{sin~} x \leq x \]
    for \(x \geq 0\).
\end{solution}


\section{The Second Fundamental Theorem: Differentiating Integrals}


\begin{theorem}[6.26, The Mean Value Theorem for Integrals]
  Suppose that \(f:[a,b]\to\mb R\) is continuous. Then there is a point \(x_0\)
  in the interval \([a,b]\) at which
  \[
    \frac{1}{b-a}\int_a^b f
      =
    f(x_0)
  .\]
\end{theorem}

\begin{proof}
    Because \(f\) is continuous on \([a,b]\), by the Extreme value theorem, 
    there exists points \(x_m, x_M \in [a,b]\) at which \(f\) obtains its 
    minimum and maximum, respectively. In other words,
    \[ f(x_m) \leq f(x) \leq f(x_M) \]
    for all \(x \in [a,b]\). Applying the monotonicity property of integrals yields
    \[ f(x_m)(b - a) = \int_a^b f(x_m) ~dx \leq \int_a^b f(x) ~dx \leq \int_a^b f(x_M) ~dx = f(x_M)(b - a) .\]
    Dividing by \(b - a\), we get
    \[ f(x_m) \leq \frac{1}{b - a} \int_a^b f(x) ~dx \leq f(x_M) .\]
    And so, by the Intermediate value theorem, there exists a point \(x_0 \in (x_m, x_M)\)
    such that \(\frac{1}{b-a} \int_a^b f = f(x_0) \).
\end{proof}


\begin{proposition}[6.27]
  Suppose that the function \(f:[a,b]\to\mb R\) is integrable. Define
  \[
    F(x) = \int_a^x f
    \text{~~for all~} x\in[a,b]
  .\]
  Then the function \(F:[a,b]\to\mb R\) is continuous.
\end{proposition}

\begin{proof}
    Since \(f\) is integrable on \([a,b]\), then by additivity, 
    \(f\) is integrable on \([a,x]\) for \(x \in [a,b]\). This
    means that \(F\) is defined on \([a,b]\). Now, \(f\) is bounded
    (since it is integrable) and so choose \(M > 0\) such that
    \( |f(x)| \leq M\) for \(x \in [a,b]\). We will show that \(F\)
    satisfies the Lipschitz condition, which implies continuity:
    \[ |F(u) - F(v)|  \leq M |u - v| \]
    for all points \(u, v \in [a,b]\). And so, let \(u, v \in [a,b]\)
    with \(u < v\). Then by additivity,
    \[ F(v) = \int_a^v f = \int_a^u f + \int_u^v f = F(u) + \int_u^v f \]
    so that
    \[ F(v) - F(u) = \int_u^v f .\]
    Then, by Lemma 6.1 (or by monotonicity...it really doesn't matter), we get
    \[ -M(v - u) \leq \int_u^v f = F(v) - F(u) \leq M(v - u) \]
    so that
    \[ |F(v) - F(u)| \leq M|u - v|. \]
    And therefore, using the \(\epsilon-\delta\) definition of continuity
    with \(\delta = \epsilon/ M\), \(F\) is continuous on \([a,b]\).
\end{proof}


\begin{theorem}[6.29, The Second Fundamental Theorem: Differentiating Integrals]
  Suppose that \(f:[a,b]\to\mb R\) is continuous. Then
  \[
    \frac{d}{dx}\left[\int_a^x f \right]
      =
    f(x)
    \text{~~for all~} x\in(a,b)
  .\]
\end{theorem}

\begin{proof}
    Define for \(x \in [a,b]\) the function 
    \[ F(x) \equiv \int_a^x f . \]
    By the previous proposition, \(F\) is defined on \([a,b]\) and continuous.
    It then suffices to show that
    \[ \lim_{x \to x_0} \frac{F(x) - F(x_0)}{x - x_0} = f(x_0) .\]
    So let \(x \in (a,b)\) be different from \(x_0\). By additivity,
    \[ F(x) = \int_a^x f = \int_a^{x_0} f + \int_{x_0}^x f = F(x_0) + \int_{x_0}^x f \]
    if \(x_0 < x\) so that \(F(x) - F(x_0) = \int_{x^0}^x f \).
    Similarly, \(F(x) - F(x_0) = - \int_{x^0}^x f \) if \(x < x_0\). Then,
    by the Mean value theorem for integrals, there exists a point \(c\) between \(x\) and \(x_0\)
    \[ \frac{1}{x - x_0} \int_{x_0}^x f = \frac{F(x) - F(x_0)}{x - x_0} = f(c) .\]
    
    By this reasoning, we construct a sequence \(\{x_n\}\) so that \(\limit x_n = x_0\).
    Then, we construct another sequence \(\{c_n\}\) as the points chosen from
    the MVT so that \(\limit c_n = x_0\), or equivalently, \(\lim_{x \to x_0} c_x = x_0\).
    Consequently, by the continuity of \(f\),
    \[ F'(x_0) = \lim_{x \to x_0} \frac{F(x) - F(x_0)}{x - x_0} = \lim_{x \to x_0} f(c_x) = f(x_0) .\]
\end{proof}


\begin{exercise}[2b]
  Suppose \(f:[0,2]\to\mb R\) is defined by
  \[
    f(x) =
    \begin{cases}
      x^2 & \text{if } 0\leq x\leq 1 \\
      x   & \text{if } 1< x\leq 2
    \end{cases}
  .\]
  Define
  \[
    F(x)=\int_a^x f(t)~dt
    \text{~~for all~} x\in[a,b]
  \]
  and find a formula for \(F(x)\) which does not involve integrals.
\end{exercise}

\begin{solution}
    If \(x \in [0,1]\), by the First Fundamental Theorem of Calculus (FTC),
    \[ F(x) = \int_0^x t^2 ~dt = \frac{x^3}{3} - \frac{0^3}{3} = \frac{x^3}{3} . \] 
    Otherwise, if \(x \in [1,2]\), then
    \[ F(x) = F(1) + \int_1^x t ~dt = \frac{1}{3} + [ \frac{x^{2}}{2} - \frac{1^2}{2} ] 
        = \frac{x^{2}}{2} - \frac{1}{6} .\]
    Notice that when \(x = 1\), both pieces of \(F\) agree. Thus,
    \[ F(x) = \begin{cases}
                    \frac{x^{3}}{3} &\text{if } x \in [0, 1] \\
                    \frac{x^{2}}{2} - \frac{1}{6} &\text{if } x \in [1, 2] \\
              \end{cases} .\]
\end{solution}


\begin{exercise}[5]
  Suppose \(f:\mb R\to\mb R\) is continuous. Define
  \[
    G(x)
      =
    \int_0^x (x-t)f(t)~dt
    \text{~~for all~} x
  .\]
  Prove that \(G''(x)=f(x)\) for all \(x\).
\end{exercise}

\begin{solution}
    Since \(f\) is continous, \(f\) is integrable so let \(F(x) = \int_0^x f(t) ~dt\). 
    Then because \( (x-t)f(t) = xf(t) - tf(t)\), it follows that
    \begin{align*} 
        \frac{d}{dx} G(x) &= \frac{d}{dx} [ \int_0^x (x - t)f(t) ~dt ] \\
                          &= \frac{d}{dx} [ x \int_0^x f(t) ~dt - \int_0^x tf(t) ~dt ] \\
                          &= \frac{d}{dx} [x F(x)] - \frac{d}{dx}[ \int_0^x tf(t) ~dt ] .\\
    \end{align*}

    Applying the product rule to the first term and the 2nd FTC to the continuous 
    function \(t f(t)\), we get
    \begin{align*}
        G'(x) &= xF'(x) + F(x) - x f(x) \\
              &= xf(x)  + F(x) - x f(x) \\
              &= F(x) .\\
    \end{align*}
    Consequently, \(G''(x) = F'(x) = f(x)\).


\end{solution}


\begin{exercise}[12]
  Suppose that \(f,g:[a,b]\to\mb R\) are continuous and that \(\alpha,\beta\)
  are real numbers. Define
  \[
    H(x)
      =
    \int_a^x[\alpha f+\beta g]-\alpha\int_a^x[f]-\beta\int_a^x[g]
    \text{~~for all~} x\in[a,b]
  .\]
  Prove that \(H(a)=0\) and \(H'(x)=0\) for all \(x\in(a,b)\).
  Use this fact and the Identity Criterion to give an alternate proof of
  Theorem 6.15 for continuous functions.
\end{exercise}

\begin{solution}
    First, we note that since \(f\) and \(g\) are continuous on \([a,b]\), so is
    \(\alpha f + \beta g\). Then, since continuity implies integrability,
    \(f\), \(g\), and \(\alpha f + \beta g\) are integrable. Next, for any 
    integrable function \(h\) on \([a,b]\), \(\int_a^a h = 0\). Consequently, 
    we have
    \[ H(a) = \int_a^a [\alpha f + \beta g] - \alpha \int_a^a [f] - \beta \int_a^a [g] 
            = 0 - \alpha 0 - \beta 0 = 0 ,\]
    
    and we apply the 2nd FTC to obtain
    \begin{align*}
        H'(x) &= \frac{d}{dx} \Bigg[ \int_a^x \alpha f + \beta g \Bigg] - \alpha \frac{d}{dx} \Bigg[ \int_a^x [f] \Bigg] - \beta \frac{d}{dx} \Bigg[ \int_a^x [g] \Bigg] \\
              &= (\alpha f + \beta g)(x) - \alpha f(x) - \beta f(x) \\
              &= \alpha f(x) + \beta g(x) - \alpha f(x) - \beta f(x) \\
              &= 0 \\
    \end{align*}

    Now, because \(H(a) = 0\), we have that
    \[ \int_a^a [\alpha f + \beta g] = \alpha \int_a^a [f] + \beta \int_a^a [g] ,\]
    and because \(H'(x)\) is identically zero, we have that
    \[ \frac{d}{dx} \Bigg[ \int_a^x \alpha f + \beta g \Bigg] = \frac{d}{dx} \Bigg[ \alpha \int_a^x [f] + \beta \int_a^x [g] \Bigg] .\]
    Therefore, by the Identity Criterion, 
    \[ \int_a^x \alpha f + \beta g = \alpha \int_a^x f + \beta \int_a^x g \]
    which proves the linearity property of integrals for continous functions.
\end{solution}



\setcounter{chapter}{9}
\chapter{The Euclidean Space \texorpdfstring{$\mb R^n$}{Rn}}


\section{The Linear Structure of \texorpdfstring{$\mb R^n$}{Rn}
and the Scalar Product}

\begin{proposition}[10.2]
  Let \(\vect u,\vect v,\vect w\in\mb R^n\)
  and \(\alpha,\beta\in\mb R\). Then both of the following hold:
  \[
    \<\vect u,\vect v\>=\<\vect v,\vect u\>
  \]
  \[
    \<\alpha\vect u+\beta\vect w, \vect v\>
      =
    \alpha\<\vect u,\vect v\>+\beta\<\vect w,\vect v\>
  \]
\end{proposition}

\begin{proof}
    Since \(\mb R\) is commutative, \(u_i v_i = v_i u_i\) for numbers
    \(u_i, v_i\). It then follows that
    \[ \<\vect u, \vect v \> = \sumi u_i v_i = \sumi v_i u_i = \<\vect v, \vect u \> .\]

    Next,
    \begin{align*} 
        \< \alpha \vect u + \beta \vect w, \vect v \> &= \sumi (\alpha u_i + \beta w_i ) v_i \\
                    &= \sumi (\alpha u_i v_i + \beta w_i v_i ) \\
                    &= \sumi (\alpha u_i v_i) + \sumi (\beta w_i v_i) \\
                    &= \alpha \sumi u_i v_i + \beta \sumi w_i v_i \\
                    &= \alpha \< \vect u, \vect v \> + \beta \< \vect w, \vect v \> \\
    \end{align*}
\end{proof}

\begin{lemma}[10.4]
  For \(\vect u,\vect v\in\mb R^n\), \(\vect u,\vect v\) are
  orthogonal if and only if
  \(\|\vect u+\vect v\|^2 =\|\vect u\|^2+\|\vect v\|^2\).
\end{lemma}

\begin{proof}
    Here, we use the definition of the norm and apply Proposition 10.2
    (taking \(\alpha = \beta = 1\)):

    \begin{align*}
         \|\vect u + \vect v \|^2 &= \< \vect u + \vect v , \vect u + \vect v \> \\
                    &= \<\vect u, \vect u + \vect v \> + \<\vect v, \vect u + \vect v \> \\
                    &= \<\vect u + \vect v, \vect u \> + \<\vect u + \vect v, \vect v \> \\
                    &= \<\vect u, \vect u \> + \<\vect v, \vect u\> + \<\vect u, \vect v\> + \<\vect v, \vect v\> \\
                    &= \| \vect u \|^2 + 2 \< \vect u, \vect v \> + \| \vect v \|^2 \\
    \end{align*}

    Since norms are nonzero, the above shows that 
    \(\|\vect u + \vect v\|^2 = \|\vect u\|^2 + \|\vect v\|^2 \)
    iff \(\< \vect u, \vect v\> = 0\), or equivalently, iff \(\vect u\) and 
    \(\vect v\) are orthogonal.
\end{proof}



\begin{lemma}[10.5]
  For \(\vect u,\vect v\in\mb R^n\) where \(\vect v\not=\vect 0\),
  define \(\lambda=\frac{\<\vect u,\vect v\>}{\<\vect v,\vect v\>}\)
  and \(\vect w=\vect u-\lambda\vect v\). Then \(\vect v,\vect w\)
  are orthogonal and \(\vect u=\vect w+\lambda\vect v\).
\end{lemma}

\begin{proof}
    Clearly, 
    \[ \vect w + \lambda \vect v = (\vect u - \lambda \vect v) + \lambda \vect v = \vect u .\]
    
    To show orthogonality,
    \begin{align*} 
        \< \vect w, \vect v \> &= \< \vect u - \lambda \vect v , \vect v \> \\
                    &= \< \vect u, \vect v\> - \lambda \<\vect v, \vect v \> \\
                    &= \< \vect u, \vect v\> - \frac{\<\vect u, \vect v\>}{\<\vect v, \vect v\>} \< \vect v, \vect v \> \\
                    &= \< \vect u, \vect v\> - \< \vect u, \vect v \> \\
                    &= 0
    \end{align*}

\end{proof}

\begin{theorem}[10.6, The Cauchy-Schwarz Inequality]
  For any two vectors \(\vect u,\vect v\in\mb R^n\),
  \[
    |\<\vect u,\vect v\>|
      \leq
    \|\vect u\|\|\vect v\|
  .\]
\end{theorem}

\begin{proof}
    If \(\vect v = \vect 0\), then 
    \[ \< \vect u, \vect v \>  = \< \vect u, \vect 0\> = \sumi u_i 0 = 0 \]
    and \(\|\vect v\| = \<\vect v, \vect v\> = 0 \). It follows that
    \[ 0 = | \< \vect u, \vect v \> | \leq \|\vect u\| \cdot \|\vect v \| = 0 .\]
    Now consider when \(\vect v \neq \vect 0\). Define 
    \(\lambda = \<\vect u, \vect v\> / \< \vect v, \vect v\>\) like in the previous
    lemma. Then \(\vect u - \lambda \vect v\) is orthogonal to \(\vect v\) (and
    consequently \(\lambda \vect v\)). Then,
    \begin{align*} 
        0 &\leq \<\vect u - \lambda \vect v, \vect u - \lambda \vect v \> \\
          &= \|\vect u\|^2 + \|\lambda \vect v\|^2 - 2\<\vect u, \vect \lambda \vect v \> \\
          &= \|\vect u\|^2 + \lambda^2 \|\vect v\|^2 - 2 \lambda \<\vect u, \vect v \> \\ 
          &= \|\vect u\|^2 + \frac{\<\vect u, \vect v\>^2}{\<\vect v, \vect v\>^2} \<\vect v, \vect v\> - 2\lambda \<\vect u, \vect v\> \\
          &= \|\vect u\|^2 + \frac{\<\vect u, \vect v\>^2}{\<\vect v, \vect v\>} - 2 \frac{\<\vect u, \vect v\>^2}{\<\vect v, \vect v\>} \\
          &= \|\vect u\|^2 - \frac{\<\vect u, \vect v\>^2}{\|\vect v\|^2}. \\
    \end{align*}

    Therefore, we have \(\<\vect u, \vect v\>^2 \leq \|\vect u\|^2 \|\vect v\|^2 \), which
    after taking the square root to both sides gives our result.
\end{proof}

\begin{theorem}[10.7, The Triangle Inequality]
  For any two vectors \(\vect u,\vect v\in\mb R^n\),
  \[
    \|\vect u+\vect v\|
      \leq
    \|\vect u\|+\|\vect v\|
  .\]
\end{theorem}

\begin{proof}
    Clearly, the inequality holds iff 
    \[ \|\vect u + \vect v\|^2 \leq (\|\vect u \| + \|\vect v\|)^2 .\]

    From earlier computation, we know
    \[ \|\vect u + \vect v\|^2 = \|\vect u\|^2 + 2\<\vect u, \vect v\> + \|\vect v\|^2 .\]

    Then, by the Cauchy-Schwarz inequality,
    \[ \|\vect u + \vect v\|^2 \leq 
        \|\vect u\|^2 + 2\|\vect u\|\|\vect v\| + \|\vect v \|^2 
        = (\|\vect u\| + \|\vect v\|)^2 \]
\end{proof}

\begin{exercise}[3]
  Show that for \(\vect u\in\mb R^n\), \(\alpha\in\mb R\):
  \begin{enumerate}[(a)]
    \item \(\|\vect u\|=0\) if and only if \(\vect u=\vect 0\).
    \item \(\|\alpha\vect u\|=|\alpha|\|\vect u\|\).
  \end{enumerate}
\end{exercise}

\begin{solution}
    (a) Suppose \(\vect u = \vect 0\), then for any \(\vect v \in \mb R^n\),
    \[ \<\vect v, \vect 0\> = \sumi v_i 0 = \sumi 0 = 0 .\]
    It follows that 
    \[ \|\vect u\| = \sqrt{\<\vect u, \vect u \>} = \sqrt{\<\vect 0, \vect 0\>} = \sqrt{0} = 0 .\]

    Conversely, suppose \(\|\vect u\| = 0\). Then \(\<\vect u, \vect u\> = \sumi u_i^2 = 0 \)
    but since square terms are nonnegative, the sum can be zero iff \(u_i = 0\) for each
    \(i \in [n]\). Consequently, \(\vect u = \vect 0\).

    (b) We show that \( \|\alpha \vect u\|^2 = \alpha^2 \|\vect u\|^2\) from which the result
    follows.
    \[ \|\alpha \vect u\|^2 = \< \alpha \vect u, \alpha \vect u\> 
        = \alpha \<\vect u, \alpha \vect u\> 
        = \alpha \< \alpha \vect u, \vect u \>
        = \alpha^2 \< \vect u, \vect u \>
    = \alpha^2 \|\vect u\|^2 \]
\end{solution}

\begin{exercise}[4]
  For \(\vect u,\vect v\in\mathbb R^n\) verify the identity
  \[
    \|\vect u-\vect v\|^2
      =
    \|\vect u\|^2+\|\vect v\|^2-2\<\vect u,\vect v\>
  .\]
\end{exercise}

\begin{solution}
    \begin{align*}
        \|\vect u - \vect v \|^2 &= \<\vect u - \vect v, \vect u - \vect v\> \\
                    &= \<\vect u + (-\vect v), \vect u + (-\vect v)\> \\
                    &= \<\vect u, \vect u + (-\vect v)\> + \<-\vect v, \vect u + (-\vect v)\> \\
                    &= \<\vect u + (-\vect v), \vect u\> - \< \vect v, \vect u + (-\vect v) \> \\
                    &= \<\vect u, \vect u\> + \< -\vect v, \vect u\> - \<\vect u + (-\vect v), \vect v \> \\
                    &= \|\vect u\|^2 - \<\vect v, \vect u\> - \<\vect u, \vect v\> - \<-\vect v, \vect v \>\\
                    &= \|\vect u\|^2 - 2\<\vect u, \vect v\> + \|\vect v\|^2 \\
    \end{align*}
\end{solution}

\begin{exercise}[9]
  Let \(\vect u\in\mb R^n\) and suppose \(\|\vect u\|<1\).
  Show that for \(\vect v\in\mb R^n\),
  \(\|\vect v-\vect u\|<1-\|\vect u\|\) implies
  \(\|\vect v\|<1\).
\end{exercise}

\begin{solution}
    By the Triangle Inequality,
    \[ \|\vect v\| = \|(\vect v - \vect u) + \vect u \| 
        \leq \|\vect v - \vect u\| + \|\vect u\|
        < (1 - \|\vect u\| ) + \|\vect u\|
    = 1 .\]
\end{solution}

\begin{exercise}[10]
  Let \(\vect u\in\mb R^n\) and \(r>0\). Suppose \(\vect v,\vect w\in\mb R^n\)
  are at a distance less than \(r\) from \(\vect u\). Prove that if
  \(0\leq t\leq 1\), then the point \(t\vect v+(1-t)\vect w\) is also
  at a distance less than \(r\) from \(\vect u\).
\end{exercise}

\begin{solution}
    \begin{align*} 
        \| [t\vect v + (1-t)\vect w] - \vect u \| &= \| t\vect v + (1-t)\vect w - t\vect u - (1-t) \vect u \| \\ 
                        &= \| t[\vect v - \vect u] + (1-t)[\vect w - \vect u] \| \\
                        &\leq \| t(\vect v - \vect u )\| + \|(1 - t)(\vect w - \vect u) \| \\
                        &= t \|\vect v - \vect u \| + (1 - t) \| \vect w - \vect u \| \\
                        &< tr + (1-t)r \\
                        &= r
    \end{align*}
\end{solution}


\section{Convergence of Sequences in \texorpdfstring{$\mb R^n$}{Rn}}

\begin{theorem}[10.9, The Componentwise Convergence Criterion]
  Let \(\{\vect u_k\}\) be a sequence in \(\mb R^n\). Then
  \(\{\vect u_k\}\) converges to \(\vect u\) if and only if
  \(\{p_i(\vect u_k)\}\) converges to \(p_i(\vect u)\) for
  all \(1\leq i\leq n\).
\end{theorem}

\begin{proof}
    Suppose \(\vect u_k \to \vect u\). For any fixed \(i \in [n]\),
    \[ 0 \leq |p_i(\vect u_k) - p_i(\vect u)| = |p_i(\vect u_k - \vect u)|  \leq \|\vect u_k - \vect u \| \]
    for all \(k\). Taking the limit of \(k\), we get
    \[ 0 \leq \lim_{k \to \infty} |p_i(\vect u_k) - p_i(\vect u)| \leq \lim_{k \to \infty} \|\vect u_k - \vect u\| = 0 .\]
    Hence, \(p_i(\vect u_k) \to p_i(\vect u)\) for all \(i \in [n]\).


    Now suppose the converse. Then \(p_i(\vect u_k - \vect u) \to 0\) for \(i \in [n]\)
    and so
    \[ \lim_{k \to \infty} \|\vect u_k - \vect u \|^2 
        = \lim_{k \to \infty} \sumi [p_i(\vect u_k - \vect u)]^2 
        = \sumi \lim_{k \to \infty} [p_i(\vect u_k - \vect u)]^2 
        = \sumi 0 = 0 . \]

    Then, since the square root function is continous, we get 
    \[ \lim_{k \to \infty} \|\vect u_k - \vect u \| = 0 .\]
    Therefore, \(\vect u_k \to \vect u\).
\end{proof}

\begin{theorem}[10.10]
  Let \(\{\vect u_k\}\), \(\{\vect v_k\}\) be sequences in \(\mb R^n\)
  such that \(\{\vect u_k\}\) converges to \(\vect u\) and
  \(\{\vect v_k\}\) converges to \(\vect v\). Then for any
  \(\alpha,\beta\in\mb R\),
  \[
    \lim_{k\to\infty}[\alpha\vect u_k+\beta\vect v_k]
      =
    \alpha\vect u+\beta\vect v
  .\]
\end{theorem}

\begin{proof}
    By the previous theorem, it suffices to show that 
    \( p_{i}(\alpha \vect u_k + \beta \vect v_k) \to p_i(\alpha \vect u + \beta \vect v) \)
    for all \(i \in [n]\). Note that
    \[ p_i(\alpha \vect u_k + \beta \vect v_k) = \alpha p_i(\vect u_k) + \beta p_i(\vect v_k) .\]
    Since \(\vect u_k \to \vect u\), \(p_i(\vect u_k) \to p_i(\vect u)\).
    Likewise, \(p_i(\vect v_k) \to p_i(\vect v)\). Hence,
    \[  p_i(\alpha \vect u_k + \beta \vect v_k) = 
        \alpha p_i(\vect u_k) + \beta p_i(\vect v_k) 
        \to \alpha p_i(\vect u) + \beta p_i(\vect v) 
    = p_i(\alpha \vect u + \beta \vect v) .\]

\end{proof}

\begin{exercise}[1]
  Let \(\{\vect u_k\}\) be a sequence in \(\mb R^n\) that converges to
  \(\vect u\). Prove the following for all \(\vect v\in\mb R^n\):
  \[
    \lim_{k\to\infty}\<\vect u_k,\vect v\>=\<\vect u,\vect v\>
  .\]
\end{exercise}

\begin{solution}
    Since \(\vect u_k \to \vect u\), \(p_i(\vect u_k) \to p_i(\vect u)\) for
    \(i \in [n]\). Then,
    \begin{align*} 
        \lim_{k \to \infty} \<\vect u_k, \vect v \> &= \lim_{k \to \infty} \sumi p_i(\vect u_k) v_i \\
                        &= \sumi [\lim_{k \to \infty} p_i(\vect u_k)] v_i \\
                        &= \sumi p_i(\vect u) v_i \\
                        &= \<\vect u, \vect v \> \\
    \end{align*}
\end{solution}

\begin{exercise}[2]
  Let \(\{\vect u_k\}\) be a sequence in \(\mb R^n\) and
  \(\vect u\in\mb R^n\). Prove that if
  \[
    \lim_{k\to\infty}\<\vect u_k,\vect v\>=\<\vect u,\vect v\>
  \]
  holds for all \(\vect v\in\mb R^n\), then \(\{\vect u_k\}\) converges
  to \(\vect u\).
\end{exercise}

\begin{solution}
    Let \(\vect e_i\) be the ith standard coordinate vector in \(\mb R^n\).
    Since \(p_i(\vect u_k) = \<\vect u_k, \vect e_i \>\), by assumption, we have for \(i \in [n]\)
    \[ \lim_{k \to \infty} p_i(\vect u_k) = 
        \lim_{k \to \infty} \<\vect u_k, \vect e_i\> = 
    \< \vect u, \vect e_i \> .\]

    From this follows
    \begin{align*} 
        \lim_{k \to \infty} \vect u_k &= \lim_{k \to \infty} \sumi p_i(\vect u_k) \vect e_i \\
                    &= \lim_{k \to \infty} \sumi \<\vect u_k, \vect e_i \> \vect e_i \\
                    &= \sumi [\lim_{k \to \infty} \<\vect u_k, \vect e_i \>] \vect e_i \\
                    &= \sumi \<\vect u, \vect e_i \> \vect e_i \\
                    &= \sumi p_i(\vect u) \vect e_i \\
                    &= \vect u \\
    \end{align*}
\end{solution}

\begin{exercise}[5]
  Let \(\{\vect u_k\}\) be a sequence in \(\mb R^n\) that converges to
  \(\vect u\) where \(\|\vect u\|= r >0\). Prove that there is an index \(K\)
  where
  \[
    \|\vect u_k\|>\frac{r}{2}
    \text{~~if~}
    k\geq K
  .\]
\end{exercise}

\begin{solution}
    We first claim that \(\vect u_k \to \vect u\) implies 
    \(\|\vect u_k \| \to \|\vect u\|\). For this, it suffices to show that
    \( \|\vect u_k \|^2 \to \|\vect u \|^2\). And so
    \begin{align*}
        \lim_{k \to \infty} \|\vect u_k\|^2 &= \lim_{k \to \infty} \<\vect u_k, \vect u_k \> \\
                                            &= \lim_{k \to \infty} \sumi p_i(\vect u_k)^2 \\
                                            &= \sumi \lim_{k \to \infty} p_i(\vect u_k)^2 \\
                                            &= \sumi p_i(\vect u)^2 \\
                                            &= \< \vect u, \vect u \> \\
                                            &= \|\vect u\|^2 \\
    \end{align*}

    Choose \(\epsilon > 0\) to be less than \(r\). By convergence, there exists 
    a natural number \(K\) such that
    \[ | \|\vect u_k\| - \|\vect u\| | < \epsilon/2 \] 
    for \(k \geq K\), or equivalently, 
    \[ \|\vect u\| - \epsilon/2 < \|\vect u_k\| < \|\vect u\| + \epsilon/2 .\]
    This implies
    \[ 2 \|\vect u_k\| > 2\|\vect u\| - \epsilon = 2r - \epsilon > 2r - r = r \]
    so that
    \[ \|\vect u_k\| > \frac{r}{2} \text{~~if~~} k \geq K .\]
\end{solution}


\section{Open Sets and Closed Sets in \texorpdfstring{$\mb R^n$}{Rn}}

\begin{remark}
    Whenever possible, I opt for a direct proof for open and
    closed sets rather than rely on DeMorgan's laws and set-theoretic
    identities. Why? I prefer the challenge!
\end{remark}

\begin{example}[10.11]
  Let \(a<b\) be in \(\mb R\). Then \(\interior(a,b]=(a,b)\).
\end{example}

\begin{proof}
    By definition, \(\interior(a,b] \subseteq (a,b]\). Hence,
    it suffices to show that \((a,b) \subset \interior(a,b]\) but that
    \(b\) is not an interior point.

    Let \(u\) be a point in \((a,b)\). Define 
    \(R = \text{min~}(u-a, b-u)\) and pick \(r < 0\) so that \(r < R\).
    Now consider some \(v \in B_r(u)\). This means that 
    \(u - r < v < u + r\). If \(R = u -a\), then \(u - a \leq b - u\)
    so that \(2u \leq a + b \). From this we get
    \[ a = u - (u - a) < u - r < v < u + r < u + (u - a) \leq u + (b - u) = b \]
    and consequently, \(v \in (a, b)\). Else, if \(R = u - b\), 
    then \(b - u \leq u - a\) and \(a + b \leq 2u\). Consequently,
    \[ a = (a + b) - b  \leq 2u - b = u - (b - u) < u - r < v < u + r < u + (b - u) = b \]
    which means \(v \in (a, b)\).
    Altogether, this shows that \((a, b) \subset \interior(a,b]\). Last,
    we consider point \(b\). For any \(r > 0\), \(B_r(b)\) is not a subset
    of \((a,b]\) since we can pick a point \(z\) such that \(b < z < b + r\).
    Therefore, \(b\) is not an interior point of \((a,b]\) and so
    \(\interior(a,b] = (a,b)\).
\end{proof}

\begin{example}[10.12]
  Let \(\mb Q\subseteq\mb R\) be the set of rational real numbers.
  Then \(\interior\mb Q=\emptyset\).
\end{example}

\begin{proof}
    Trivially, \(\emptyset \subset \interior \mb Q\), so it 
    suffices to show that no point in \(\mb Q\) is interior.

    Let \(q\) be any rational number, and choose any \(r > 0\)
    for an open ball \(B_r(q)\) in \(\mb R\). Because the irrational
    numbers are dense in \(\mb R\) and \(B_r(q) = (q - r, q + r)\)
    (an open interval), there exists an irrational number \(s\)
    in \(B_r(q)\). Consequently, \(B_r(q) \not\subset \mb Q\).
    Hence \(q\) is not an interior point and \(\interior \mb Q = \emptyset\).
\end{proof}

\begin{proposition}[10.13]
  Every open ball \(B_r(\vect u)\) in \(\mb R^n\) is open.
\end{proposition}

\begin{proof}
    Let \(\vect v \in B_r(\vect u)\). Then, by definition
    \(dist(\vect u , \vect v) < r\) and so we define 
    \(R = r - dist(\vect u, \vect v) > 0 \). Now consider
    any \(w \in B_{R}(\vect v)\). Then by the Triangle
    Inequality
    \[ dist(\vect w, \vect u) \leq dist(\vect w, \vect v) + dist(\vect v, \vect u)
        < R + dist(\vect v, \vect u) 
        = [r - dist(\vect u, \vect v)] + dist(\vect u, \vect v) 
    = r \]
    which shows that \(B_{R}(\vect v) \subseteq B_{r}(\vect u)\).

    Hence, every point in \(B_r(\vect u)\) is interior, and so
    \(B_r(\vect u)\) is open.
\end{proof}

\begin{example}[10.14]
  Let \(a<b\) be in \(\mb R\). Then \([a,b]\) is closed.
\end{example}

\begin{proof}
    This follows from Theorem 2.22 and by definition of 
    a closed set. As a brief summary, we show that for 
    \(c_n \to c \) in \([a,b]\), we must have
    \(\limit [b - c_n] = b - c\) and \(\limit [c_n - a] = c - a\).
    Then since \(b - c \geq 0 \) and \(c - a \geq 0\), we
    have by transitivity thant \(b \geq c \geq a \) and so
    \(c \in [a,b]\).
\end{proof}

\begin{example}[10.15]
  The set
  \[
    [-1,1]\times[-1,1]
      =
    \{(x,y)\in\mb R^2:-1\leq x\leq 1 \text{~and~} -1\leq y\leq 1\}
  \]
  is closed in \(\mb R^2\).
\end{example}

\begin{proof}
    Let \(\{(x_n, y_n)\}\) be a sequence in \([-1,1]\times[-1,1]\) that
    converges to \((x,y)\). By the Componentwise Convergence Criterion,
    we must have that \(x_n \to x\) and \(y_n \to y\). Now since 
    \(\{x_n\}, \{y_n\}\) are sequences in \([-1, 1]\) and \([-1,1]\)
    is closed in \(\mb R\) (by the previous example), it follows that \(x, y \in [-1,1]\).
    Consequently, \((x,y) \in [-1,1] \times [-1,1]\) and so this
    set is closed in \(\mb R^2\).
\end{proof}

\begin{theorem}[10.16, The Complementing Characterization]
  A subset \(A\subseteq\mb R^n\) is open if and only if its complement
  \(\mb R^n\setminus A\) is closed.
\end{theorem}

\begin{proof}
    From now on, we use the notation \(A^c\) for \(\mb R^n \setminus A\).
    Suppose \(A\) is open, and let \(\{\vect x_n\}\) be a sequence in
    \(A^c\) that converges to point \(\vect x\). Suppose on the contrary
    that \(\vect x \not\in A^c\). Then \(\vect x \in A\). Because \(A\)
    is open, there exists an open ball \(B_r(\vect x) \subset A\). However
    this means that no point in the sequence \(\{\vect x_n\}\) lies in
    \(B_r(\vect x)\); this contradicts our assumption that 
    \(\vect x_n \to \vect x\). Hence \(\vect x \in A^c\) and so \(A^c\)
    is closed.

    Conversely, suppose \(A^c\) is closed, and let \(\vect y \in A\).
    Suppose on the contrary that \(\vect y\) is not an interior point 
    of \(A\). Then, every open ball around \(\vect y\) intersects
    \(A^c\) at a point distinct from \(\vect y\). From this, we can 
    construct the following sequence: for each natural number \(k\),
    pick a point \(\vect y_k\) from \(B_{\frac{1}{k}}(\vect y) \cap A^c\).
    Then, because \(\frac{1}{k} \to 0\) as \(k \to \infty\), we
    have that \(\vect y_k \to \vect y\). However, this means that
    \(\vect y_k\) is a convergent sequence in \(A^c\). By assumption,
    \(A^c\) is closed and so \(\vect y \in A^c\); this is a contradiction.
    Hence, \(\vect y\) is an interior point of \(A\). Thus, \(A\)
    is open.

\end{proof}

\begin{proposition}[10.17.i]
  The union of a collection of open subsets of \(\mb R^n\) is open.
\end{proposition}

\begin{proof}
    Let \(U = \bigcup_{\alpha \in I} U_\alpha\) where \(\{U_\alpha\}\) is a collection
    of open subsets of \(\mb R^n\). We show that every point \(\vect x\)
    in \(U\) is interior. Since \(\vect x \in U\), \(\vect x \in U_\alpha\)
    for some \(\alpha\). Because \(U_\alpha\) is open, there exists \(r > 0\)
    such that \(B_r(\vect x) \subset U_\alpha\). Then because
    \(U_\alpha \subset U\), we have that \(B_r(\vect x) \subset U\).
    Hence, \(U\) is open.
\end{proof}

\begin{proposition}[10.17.ii]
  The intersection of a collection of closed subsets of \(\mb R^n\) is closed.
\end{proposition}

\begin{proof}
    Let \(C = \bigcap_{\alpha \in I} C_\alpha\) where \(\{C_\alpha\}\) is
    a collection of closed subsets of \(\mb R^n\). Let \(\{\vect x_n\}\)
    be a sequence in \(C\) that converges to \(\vect x\). By definition,
    this sequence \(\{\vect x_n\}\) is in every \(C_\alpha\). Since 
    each of these sets is closed, we have that \(\vect x \in C_\alpha\)
    for all \(\alpha\). Hence \(\vect x \in C\). Thus \(C\) is closed.
\end{proof}

\begin{proposition}[10.18.i]
  The intersection of a finite collection of
  open subsets of \(\mb R^n\) is open.
\end{proposition}

\begin{proof}
    Let \(U = U_1 \cap U_2 \cap ... \cap U_n\) be the intersection
    of finitely many open subsets \(U_i\) of \(\mb R^n\).
    Let \(\vect x \in U\). Then \(\vect x \in U_i\) for all \(i \in [n]\).
    Since each of these \(U_i\) is open, there exists \(r_i > 0\)
    where \(B_{r_i}(\vect x) \subset U_i\). Define 
    \(r = \text{min~}\{r_1, r_2, ..., r_n\}\). It immediately
    follows that \(B_{r}(\vect u) \subset B_{r_i}(\vect u) \subset U_i\)
    for \(i \in [n]\). Thus, \(B_{r}(\vect u) \in U\) and so \(U\)
    is open.
\end{proof}

\begin{proposition}[10.18.ii]
  The union of a finite collection of
  closed subsets of \(\mb R^n\) is closed.
\end{proposition}

\begin{proof}
    Let \(C = C_1 \cup C_2 \cup ... \cup C_n \) be the union
    of finitely many closed subsets \(C_i\) in \(\mb R^n\).
    Then let \(\{\vect x_n\}\) be a sequence in \(C\) that
    converges to \(\vect x\). By definition, \(\vect x_n \in C_i\)
    for some \(i\) for every \(n\). Since \(n\) is finite, it
    follows that there exists some particular \(j \in [n]\) for which
    \(\{\vect x_n\} \cap C_j \) is nonempty and countable (not finite). 
    To put it another way, the mapping \(k \mapsto nk\) is an injection
    on the natural numbers and so by the Pigeonhole Principle, at
    least one intersection must contain a countable number of elements.
    Anyway, this intersection is a subsequence \(\{\vect x_{n_j}\}\) of 
    \(\{\vect x_n\}\). Then clearly \(\vect x_{n_j} \to \vect x\). 
    Because \(C_j\) is closed, \(\vect x \in C_j\) and so \(\vect x \in C\).
\end{proof}

\begin{proposition}[10.19.i]
  \(A\subseteq\mb R^n\) is open if and only if \(A\cap\boundary A=\emptyset\).
\end{proposition}

\begin{proof}
    Suppose \(A\) is open, and let \(\vect x \in A \cap \boundary A\). 
    Because \(\vect x \in A\), there exists \(r > 0\) such that 
    \(B_{r}(\vect x) \subset A\). However, this implies that 
    \(B_{r}(\vect x) \cap \boundary A = \emptyset\) because
    \(B_{r}(\vect x)\) does not intersect \(A^c\). Consequently,
    \(\vect x \not\in \boundary A\), a contradiction. Therefore,
    \(A \cap \boundary A = \emptyset \).

    Conversely, suppose \(A \cap \boundary A = \emptyset\) and
    let \(\vect x \in A\). By assumption, \(\vect x \not\in \boundary A\)
    and so, by definition, there exists an open ball \(B_r(\vect x)\)
    that does not intersection \(A^c\). Hence \(B_r(\vect x) \subset A\)
    and so \(A\) is open.
\end{proof}

\begin{proposition}[10.19.ii]
  \(A\subseteq\mb R^n\) is closed if and only if \(\boundary A\subseteq A\).
\end{proposition}

\begin{proof}
    Suppose \(A\) is closed. Let \(\vect x \in \boundary A\). Then
    every open ball around \(\vect x\) intersects \(A\). In particular,
    we form a sequence of points \(\vect x_k \) in \(A\) taken from
    \(B_{1/k}(\vect x)\). Then \(\vect x_k \to \vect x\). Since \(A\)
    is closed, we have that \(\vect x \in A\). Thus, \(\boundary A \subseteq A\).

    Now suppose the converse. We show that \(A^c\) is open. Let \(\vect x \in A^c\).
    Then \(\vect x \not\in \boundary A \), which means there exists an open
    ball \(B_r(\vect x)\) contained in \(A^c\). Hence \(A^c\) is open. By 
    Theorem 10.16, this means its complement is closed, which is \(A\).
\end{proof}

\begin{exercise}[2]
  Determine which of the following subsets of \(\mb R^2\) are open, closed,
  neither, or both.
  \begin{enumerate}[(a)]
    \item \(\{(x,y):x^2>y\}\)
    \item \(\{(x,y):x^2+y^2=1\}\)
    \item \(\{(x,y):x \text{~is rational}\}\)
    \item \(\{(x,y):x\geq 0,~y\geq0\}\)
  \end{enumerate}
\end{exercise}
\begin{solution}
  \begin{enumerate}[(a)]
    \item
    \item
    \item
    \item
  \end{enumerate}
\end{solution}

\begin{exercise}[3]
  Let \(r>0\) and \(O=\{\vect u\in\mb R^n:\|\vect u\|>r\}\). Prove that \(O\)
  is open.
\end{exercise}
\begin{solution}

\end{solution}

\begin{exercise}[7a]
  Show that \(A\subseteq\mb R^n\) is open if and only if
  \[
    \vect w + A = \{\vect w+\vect u:\vect u\in A\}
  \]
  is open for all \(\vect w\in\mb R^n\).
\end{exercise}
\begin{solution}

\end{solution}

\begin{exercise}[12]
  For \(A\subseteq\mb R^n\), denote its closure by
  \[
    \closure A = \interior A \cup \boundary A
  .\]
  Prove that \(A\subseteq \closure A\). Then prove that
  \(A=\closure A\) if and only if \(A\) is closed.
\end{exercise}
\begin{solution}

\end{solution}





\chapter{Continuity, Compactness, and Connectedness}

\section{Continuous Functions and Mappings}


\end{document}


